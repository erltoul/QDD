Modified make commands :

./make.sh (./make.sh 1) (./make.sh 1 netlib) 
Compile with netlib fftpack, sequential. The executable is called tdks_v000000_Mono_NETLIB.bin (000000 is the version number)

./make.sh np (./make.sh np netlib)
Compile with netlib fftpack, parallel with np proc 

./make.sh 1 fftw
Compile with FFTW functions (need FFTW 3.3.2), sequential (parallel has not been test) The executable is called tdks_v000000_Mono_FFTW.bin

Some word on wisdom (from the FFTW manual):

"FFTW implements a method for saving plans to disk and restoring them. In fact, what
FFTW does is more general than just saving and loading plans. The mechanism is called
wisdom. [...]
Plans created with the FFTW_MEASURE, FFTW_PATIENT, or FFTW_EXHAUSTIVE options produce
near-optimal FFT performance, but may require a long time to compute because FFTW
must measure the runtime of many possible plans and select the best one. This setup is
designed for the situations where so many transforms of the same size must be computed
that the start-up time is irrelevant. For short initialization times, but slower transforms,
we have provided FFTW_ESTIMATE. The wisdom mechanism is a way to get the best of both
worlds: you compute a good plan once, save it to disk, and later reload it as many times as
necessary. The wisdom mechanism can actually save and reload many plans at once, not
just one.
Whenever you create a plan, the FFTW planner accumulates wisdom, which is information
sufficient to reconstruct the plan. After planning, you can save this information to disk
[...]
The next time you run the program, you can restore the wisdom and then recreate the plan using
the same flags as before.
Wisdom is automatically used for any size to which it is applicable, as long as the planner
flags are not more “patient” than those with which the wisdom was created. For example,
wisdom created with FFTW_MEASURE can be used if you later plan with FFTW_ESTIMATE or
FFTW_MEASURE, but not with FFTW_PATIENT."

In the code, FFTW_EXHAUSTIVE is used, and wisdom is stored in wisdom_fftw.dat file.
The first time the code is launched with FFTW activated, it will warn you that wisdom_fftw.dat does not exist and it will create it.
So for each new FFT grid size, the program will compute a good plan (it could take a long time), and then it will re-use it every time the program is launched with such grid size. A wisdom file can be exported from one computer to another, but the optimal way to do FFT is depending of the architecture.
