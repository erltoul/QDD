\begin{tabular}{|p{3.5cm}|p{11.2cm}|}
	\hline
	\texttt{jrtaint} & interval for performing an RTA step. Set to value $n$, where $n$ is an integer, an RTA step will occur every $n$th iteration. Set to $n=0$ to disable RTA\\
	\hline
	\texttt{rtamu} & \\
	\hline
	\texttt{rtamuj} & \\
	\hline
	\texttt{rtasumvar2max} & \\
	\hline
	\texttt{rtaeps} & \\
	\hline
	\texttt{rtae0dmp} & \\
	\hline
	\texttt{rtasigee} & \\
	\hline
	\texttt{rtars} & \\
	\hline
	\texttt{rtatempinit} & \\
	\hline
\end{tabular}


The {\tt NAMELIST dynamic} contains the following variables
used in the RTA procedure:\PGRfoot{We should define a new {\tt
    NAMELIST} and shift the RTA variables to there.}
\begin{description}
\item[\tt jrtaint:]
   Modulus for calling the RTA subroutine, i.e., nr. of TDLDA steps
   per one RTA step. Course time step $\Delta t$ for RTA and fine
   time step for TDLDA {\tt dt1} are related as
   $\Delta t=${\tt jrtaint}$*${\tt dt1}.
\item[\tt rtamu:]
   Parameter $\mu$ in front of the quadratic density constraint in the
   DCMF Hamiltonian (\ref{eq:hDCMF}).
\item[\tt rtamuj:]
   Parameter $\mu_j$ in front of the quadratic current constraint in the
   DCMF Hamiltonian (\ref{eq:hDCMF}).
\item[\tt rtasumvar2max:] Termination criterion $\epsilon_0$ in the
   RTA step as used
   in figure \ref{fig:summaryDCMF}. 
\item[\tt rtaeps:]
  Step size $\delta$ in the damping operator \PGRcomm{(cross ref to be
    defined)} $\mathcal{D}$ for the RTA step.
\item[\tt rtae0dmp:]
  Energy offset $E_00$ in the damping operator \PGRcomm{(cross ref to be
    defined)} $\mathcal{D}$ for the RTA step.
\item[\tt rtasigee:]
  In medium $e^-$-$e^-$ cross section used for the relaxation time 
  (\ref{eq:relaxtime}).
\item[\tt rtars:]
  Effective  Wigner-Seitz radius $r_s$ used for the relaxation time 
  (\ref{eq:relaxtime}).
\item[\tt rtatempinit:]
  The value {\tt rtatempinit}/10 is used as lower value for the search
  of temperature in {\tt SUBROUTINE ferm1}.
\item[\tt rtaforcetemperature:]
  \PGRcomm{Seems to be obsolete?}
\end{description}
