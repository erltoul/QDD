\documentclass[final,1p]{elsarticle}
\usepackage{amssymb}
\usepackage{stackrel}
\usepackage{amsfonts}
\usepackage{amsmath}
\usepackage[english]{babel}
\usepackage{graphicx}
\usepackage{float}
\usepackage{rotating}
\usepackage{color}
%\usepackage{ulem}
\usepackage{tikz}
\usepackage{epic}
% \usepackage{caption,subcaption}
%\usepackage{subeqnarray}
 

\newcommand\encircle[1]{%
  \tikz[baseline=(X.base)] 
    \node (X) [draw, shape=circle, inner sep=0] {\strut #1};}

\newcommand{\I}{\mathrm{i}}

\newcommand{\PGR}[1]{{\color{blue} #1}}
\newcommand{\PGRcomm}[1]{{\color{blue}\small PGR2all: #1}}
\newcommand{\PGRfoot}[1]{{\color{blue}\footnote{\color{blue} #1}}}
%\newcommand{\ES}[1]{{\color{red} #1}}
%\newcommand{\ESfoot}[1]{{\color{red}\footnote{\color{red} #1}}}
\newcommand{\bmu}{{\mbox{\boldmath{$\mu$}}}}

\newcounter{denselistcounter}
\newenvironment{denselist}[1]
{ \begin{list}
  {#1{denselistcounter})}{\usecounter{denselistcounter}
  \setlength{\topsep}{-0pt}
  \setlength{\partopsep}{-0pt}
  \setlength{\itemsep}{-0pt}
  \setlength{\parsep}{-0pt}
  \setlength{\labelwidth}{6pt}
  \setlength{\labelsep}{4pt}
  \setlength{\leftmargin}{20pt}
  \setlength{\rightmargin}{20pt}
  }
}
{\end{list}}

\unitlength 1mm
\thicklines

\begin{document}


\begin{frontmatter}

\title{Documentationm for the RTA code in 3D}

\author{F.~Coppens$^a$}
\author{M.~Vincendon$^a$}
\author{P.-G.~Reinhard$^c$}
\author{E.~Suraud$^{a,b}$}
\cortext[author]{Corresponding author:
  coppens@irsamc.ups-tlse.fr} 
\address{$^a$Universit\'e de Toulouse; UPS; Laboratoire de Physique
             Th\'{e}orique, IRSAMC; F-31062 Toulouse Cedex, France}
\address{$^b$Laboratoire de Physique Th\'eorique, Universit\'e Paul
  Sabatier, CNRS, F-31062 Toulouse C\'edex, France}
\address{$^c$Institut f{\"u}r Theoretische Physik, Universit{\"a}t
  Erlangen, D-91058 Erlangen, Germany}

\date{\today}
\begin{abstract}
We present a RTA+TDLDA code on a cartesian 3D grid ....
\end{abstract}

\begin{keyword}
%\PACS{05.30.Fk,31.70.Hq,34.10.+x,36.40.Cg}
electron-electron collisions, electronic disspation, time-dependent
density functional theory, metal cluster, plasmon, electron emission
\end{keyword}
\end{frontmatter}

\newpage


\section{The structure of the RTA package in {\tt rta.F90}}

\subsection{The calling tree}

Here is an oversight over the tree structure of the RTA routines.
Those subroutines contained in  {\tt rta.F90} are explained in detail
in section \ref{eq:details}. Subroutines coming from the TDLDA package
or external sources are marked\PGRfoot{The {\tt HEigensystem} seems
  copied from some library. This could cause copyright problems if we
  publish the code. Is it from BLAS/LINPACK? Then we could
replace the Fortran source by a library call.}
\\
\centerline{\fbox{
\begin{picture}(122,159)(5,-155)
\put(10,0){\mbox{\tt RTA}}
\put(11,-1){\line(0,-1){150}}
   \put(20,-4){\mbox{\tt srhomat}}
   \put(11,-3){\line(1,0){8}}
   \put(21,-5){\line(0,-1){6}}
      \put(30,-8){\mbox{\tt scalar}}
      \put(21,-7){\line(1,0){8}}
      \put(30,-12){\mbox{\tt cdiagmat}}
      \put(21,-11){\line(1,0){8}}
      \put(31,-13){\line(0,-1){2}}
         \put(40,-16){\mbox{\tt HEigensystem}}
         \put(31,-15){\line(1,0){8}}
             \put(90,-16){\mbox{$\leftrightarrow$ library routine}}
   \put(20,-20){\mbox{\tt eqstate}}
   \put(21,-21){\line(0,-1){74}}
   \put(11,-19){\line(1,0){8}}
      \put(30,-24){\mbox{\tt calcrhotot}}
      \put(21,-23){\line(1,0){8}}
      \put(30,-28){\mbox{\tt calc\_current}}
      \put(21,-27){\line(1,0){8}}
      \put(30,-32){\mbox{\tt calc\_Eref}}
      \put(21,-31){\line(1,0){8}}
      \put(30,-36){\mbox{\tt fermi1}}
      \put(31,-37){\line(0,-1){2}}
      \put(21,-35){\line(1,0){8}}
         \put(40,-40){\mbox{\tt occT1}}
         \put(31,-39){\line(1,0){8}}
      \put(30,-44){\mbox{\tt calc\_psi1}}
      \put(31,-45){\line(0,-1){38}}
      \put(21,-43){\line(1,0){8}}
         \put(40,-48){\mbox{\tt calc\_hamiltonien}}
         \put(31,-47){\line(1,0){8}}
         \put(40,-52){\mbox{\tt calc\_var}}
         \put(41,-53){\line(0,-1){2}}
            \put(50,-56){\mbox{\tt cproject}}
            \put(41,-55){\line(1,0){8}}
            \put(90,-56){\mbox{\PGR{$\leftrightarrow$ \footnotesize obsolete?}}}
         \put(40,-60){\mbox{\tt calcrhotot}}
         \put(31,-59){\line(1,0){8}}
         \put(40,-64){\mbox{\tt calc\_current}}
         \put(31,-63){\line(1,0){8}}
         \put(40,-68){\mbox{\tt cschmidt}}
         \put(31,-67){\line(1,0){8}}
             \put(90,-68){\mbox{$\leftrightarrow$ TDLDA package}}
         \put(40,-72){\mbox{\tt calc\_ekin}}
         \put(31,-71){\line(1,0){8}}
             \put(90,-72){\mbox{$\leftrightarrow$ TDLDA package}}
         \put(40,-76){\mbox{\tt nonlocalc}}
         \put(31,-75){\line(1,0){8}}
             \put(90,-76){\mbox{$\leftrightarrow$ TDLDA package}}
         \put(40,-80){\mbox{\tt fftf}}
         \put(31,-79){\line(1,0){8}}
             \put(90,-80){\mbox{$\leftrightarrow$ FFTW3 package}}
         \put(40,-84){\mbox{\tt fftback}}
         \put(31,-83){\line(1,0){8}}
             \put(90,-84){\mbox{$\leftrightarrow$ FFTW3 package}}
      \put(30,-88){\mbox{\tt coul\_mfield}}
      \put(21,-87){\line(1,0){8}}
             \put(90,-88){\mbox{$\leftrightarrow$ TDLDA package}}
      \put(30,-92){\mbox{\tt dyn\_mfield}}
      \put(21,-91){\line(1,0){8}}
             \put(90,-92){\mbox{$\leftrightarrow$ TDLDA package}}
      \put(30,-96){\mbox{\tt info}}
      \put(21,-95){\line(1,0){8}}
             \put(90,-96){\mbox{$\leftrightarrow$ TDLDA package}}
   \put(20,-100){\mbox{\tt occupT0}}
   \put(21,-101){\line(0,-1){2}}
   \put(11,-99){\line(1,0){8}}
      \put(30,-104){\mbox{\tt indexx}}
      \put(21,-103){\line(1,0){8}}
   \put(20,-108){\mbox{\tt calcrhoeq}}
   \put(21,-109){\line(0,-1){10}}
   \put(11,-107){\line(1,0){8}}
      \put(30,-112){\mbox{\tt cdiagmat}}
      \put(31,-113){\line(0,-1){2}}
      \put(21,-111){\line(1,0){8}}
         \put(40,-116){\mbox{\tt HEigensystem}}
         \put(31,-115){\line(1,0){8}}
             \put(90,-116){\mbox{$\leftrightarrow$ library routine}}
      \put(30,-120){\mbox{\tt  indexx}}
      \put(21,-119){\line(1,0){8}}
   \put(20,-124){\mbox{\tt calc\_Eref}}
   \put(11,-123){\line(1,0){8}}
   \put(20,-128){\mbox{\tt CorrectEnergy2}}
   \put(11,-127){\line(1,0){8}}
   \put(20,-132){\mbox{\tt OccupPerSpin}}
   \put(11,-131){\line(1,0){8}}
   \put(20,-136){\mbox{\tt temperature}}
   \put(11,-135){\line(1,0){8}}
   \put(21,-137){\line(0,-1){2}}
      \put(30,-140){\mbox{\tt lmdif1}}
      \put(21,-139){\line(1,0){8}}
   \put(20,-144){\mbox{\tt dyn\_mfield}}
   \put(11,-143){\line(1,0){8}}
             \put(90,-144){\mbox{$\leftrightarrow$ TDLDA package}}
   \put(20,-148){\mbox{\tt info}}
   \put(11,-147){\line(1,0){8}}
             \put(90,-148){\mbox{$\leftrightarrow$ TDLDA package}}
   \put(20,-152){\mbox{\tt analyze\_elect}}
   \put(11,-151){\line(1,0){8}}
             \put(90,-152){\mbox{$\leftrightarrow$ TDLDA package}}
\end{picture}
}}

\subsection{The subroutines in detail}
\label{eq:details}

\subsubsection*{\tt SUBROUTINE rta(psi,aloc,rho,iterat)}
\begin{tabular}{lcl}
 {\tt iterat} & in & external iteration number (TDLDA time step)\\
 {\tt psi(1:kdfull2,1:kstate)} & in/out& set of s.p. wavefunctions\\
 {\tt rho(1:2*kdfull2)}& in/out & local densities for spin up and down\\
 {\tt aloc(1:2*kdfull2)} & in/out& local potentials for spin up and down\\
\end{tabular}
\\[4pt]
Basic RTA routine performing density constrained mean-field (DCMF)
iterations, energy adjustment, admixing of local equilibrium states by
calls to subroutines (see calling tree).



\subsubsection*{\tt SUBROUTINE calcrhoeq(psiorthloc,psieqloc,psiloc,occuporthloc,occuploc,nstateloc)}
\begin{tabular}{lcl}
 {\tt nstateloc} & in & number of s.p. states in spin block\\
 {\tt psiorthloc(1:kdfull2,1:nstateloc),} & in & set of TDLDA
 wavefunctions (natural orbitals)\\
 {\tt occuporthloc(1:nstateloc)} & in & occupations of TDLDA states\\
 {\tt psieqloc(1:kdfull2,1:nstateloc)} & in& set of local-equilibrium wavefunctions\\
 {\tt psiloc(1:kdfull2,1:nstateloc)} & out& set of final mixed wavefunctions\\
 {\tt occuploc(1:nstateloc)} & in/out & \\
\end{tabular}
\\[4pt]
Encapsulated in {\tt SUBROUTINE rta}. Performs the mixing of TDLDA
states with local-equilibrium state according to relaxation rate
for one spin block. The mixed densty matrix is expanded in a
representation by both sets of s.p. states. 



\subsubsection*{\tt SUBROUTINE calc\_Eref(occup,ispin,Ei,Eref)}
\begin{tabular}{lcl}
 {\tt occup(1:nstate)} & in & occupation number for s.p. states.\\
 {\tt ispin(1:nstate)} & in & spin assignement for s.p. states.\\
 {\tt Ei(1:nstate)} & in & spin assignement for s.p. states.\\
 {\tt Eref(1:2)} & out & sum of s.p. energies per spin.\\
\end{tabular}
\\
Computes the weighted sum of s.p. energies as reference energy
for DCMF. The sum is accumulated for each spin separately.


\subsubsection*{\tt SUBROUTINE fermi1(ekmod,eref,occup,ispinact,T0i,T1i,T2,mu)}
\begin{tabular}{lcl}
 {\tt ekmod(1:kstate)} & in & given s.p. energies, spin up block first, then
 spin down\\
 {\tt eref}& in & reference energy = wanted sum of s.p. energies\\
 {\tt ispinact}& in & spin for which routine is run\\
 {\tt T0i, T1i}& in & lower and upper temperature for search\\
 {\tt occup(1:kstate)}& in/out & occupation numbers, spin block-wise\\
 {\tt T2} & out & final temperature for which Fermi distribution
 matches {\tt eref} \\
 {\tt mu} & out & final chemical potential\\
\end{tabular}
\\[4pt]
Determines thermal Fermi occupation such that given sum of
s.p. energies {\tt eref} and particle number is matched. Is done for
each spin separately. Solution scheme is bracketing. Refers to 
{\tt SUBROUTINE OccT1} while iterating temperatur {\tt T2}.
\\
\PGRcomm{Nr. of spin-up/spin-down states comes through {\tt m\_params}. 
We should protocol all such entries. First step is to augment each
{\tt USE} by {\tt ONLY} such that the explicitely communicated
variables becomes visible. Important variables may then be listed
explicitely.
}
\\
\PGRcomm{Routine requires that arrays are sorted in continuous blocks of
  spin. Do we have an initial check for that? And we need to address
  that in the general part which explains the layout of arrays.}

\subsubsection*{\tt SUBROUTINE OccT1(occrefloc,enerloc,Etotloc,muloc,occtotloc,n,T,occuploc)}
\begin{tabular}{lcl}
 {\tt enerloc(1:n)} & in & s.p. energies for actual spin\\
 {\tt n} & in & number of s.p. states treated here\\
 {\tt T} & in & temperature\\
 {\tt occrefloc} & in & wanted total number of particles \\
 {\tt occuploc(1:n)} & out & thermal occupation numbers for given {\tt
   T} and s.p. energies\\
 {\tt muloc} & out & chemical potential (Fermi energy)\\
 {\tt occtotloc} & out& final total number of particles \\
 {\tt Etotloc} & out& sum of s.p. energies \\
\end{tabular}
\\[4pt]
Determines by bracketing chemical potential {\tt muloc} for given array of
s.p. energies, temperature {\tt T}, and wanted number of particles
{\tt occrefloc} with precision {\tt 1D-12}. Delivers with it
thermal occupation numbers and corresponding total particle number and
sum of s.p. energies.
\\
\PGRcomm{This routine is specific to {/tt SUBROUTINE ferm1}. Could we
encapsulate it by a {/tt CONTAINS}?}

\subsubsection*{\tt SUBROUTINE
  Calc\_psi1(psi1,aloc,rhotot0,rhototloc,curr0,curr1,j,lambda,mu,lambdaj,muj,sumvar2,eal,ekmod)}
\noindent
combined with encapsulated {\tt SUBROUTINE calc\_hamiltonien}.
\\
\begin{tabular}{lcl}
 {\tt j} & in & number of DCMF iteration, used here for print\\
 {\tt lambda(1:kdfull2,1:2)} & in & Lagrange parameter for density for
 spin up\&down\\
 {\tt lambdaj(1:kdfull2,1:3)} & in & Lagrange parameter for current\\
 {\tt mu, muj} & in & driving parameter for augmented Lagrangian\\
 {\tt aloc(1:2*kdfull2)} & in & local potentials for spin up and down\\
 {\tt rhoto0(1:kdfull2,1:2)} & in & initial density \PGRcomm{not used ??}\\
 {\tt curr0(1:kdfull2,1:3)} & in & wanted current \\
 {\tt psi1(1:kdfull2,1:kstate)} & in/out & set of s.p. wavefunctions iterated\\
 {\tt rhototloc(1:kdfull2,1:2)} & out & actual density according to {\tt psi1}\\
 {\tt curr1(1:kdfull2,1:3)} & out & actual current from {\tt psi1}\\
 {\tt ekmod(1:nstate)} & out & final s.p. energies\\
 {\tt eal} & out & final sum of s.p. energies\\
 {\tt sumvar2} & out & variance of s.p. energies\\
\end{tabular}
\\[4pt]
Performs one damped gradient step of with density \& current
constrained Hamiltonian.
\\
\PGRcomm{The density array distinguishes spin up/down while the
  current array does not. Reason?}
\\
PGRcomm{The IN \& OUT assignments in this subroutine have to be updated.}



\subsubsection*{\tt SUBROUTINE eqstate(psi,aloc,rho,psi1,occuporth,iterat)}
\begin{tabular}{lcl}
 {\tt iterat} & in & actual iteration number (for printing)\\
 {\tt psi(1:kdfull2,1:kstate)} & in & initial set of
 s.p. wavefunctions\\
 {\tt psi1(1:kdfull2,1:kstate)} & out & final set of s.p. wavefunctions\\
 {\tt aloc(1:2*kdfull2)} & in/out & local part of potential, spin
 up/down stacked in blocks\\
 {\tt rho(1:2*kdfull2)} & in & initial density, spin
 up/down stacked in blocks \\
 {\tt occuporth(1:kstate)} & in & occupation numbers for {\tt psi} and
 still the same for {\tt psi1}.\\
\end{tabular}
\\[4pt]
DCMF iterations by reapeatedly calling {\tt Calc\_psi1},
updating Lagrangian parameters for density \& current constraints, and
occassionally tuning temperature to achieve correct energy. The
latter is done by calling {\tt fermi1}. The local potential
is kept constant during DCMF iteration and updated only at the very end.
\\
\PGRcomm{Fetches nr. of spin up/down from {\tt m\_params}.}
\\
\PGRcomm{Lagrange parameters are started from scratch. May it be
  faster to recycle the previous Lagrange parameters?
}
\\
\PGRcomm{Density {\tt rho} is entered via list and still recomputed
as {\tt rhotot0}. Unnecessary doubling?}


\subsubsection*{\tt SUBROUTINE OccupT0(occloc,esploc,Estar)}
\begin{tabular}{lcl}
 {\tt esploc(1:nstate)} & in & given s.p. energies\\
 {\tt occloc(1:nstate)} & in & given occupation numbers\\
 {\tt Estar} & out & excitation energy relative to T=0 distribution\\
\end{tabular}
\\[4pt]
Computes thermal excitation energy as difference of actual energy to
the energy obtained by Fermi distribution for $T=0$. The latter
distributions  is computed for the given s.p. energies which are the
same as used for the thermal state.



\subsubsection*{\tt SUBROUTINE calcrhotot(rho,q0)}
\begin{tabular}{lcl}
 {\tt q0(1:kdfull2,1:kstate) } & in & set of s.p. wavefunctions for
 which density is accumulated\\
 {\tt rho(kdfull2,2)} & out & resulting density\\
\end{tabular}
\\[4pt]
Computes local density for set of wavefunctions {\tt q0}. Note that
two crucial information is communicated via module {\tt params}, namely
{\tt occup}, the array of occupation numbers, {\tt ispin} the
array assigning spin top each s.p. state, and {\tt nstate}, the number
of s.p. states.
\\
\PGRcomm{Exploiting the sorting of spin in blocks of s.p. states, we
  could rewrite the code with to {\tt SUM} statements.}


\subsubsection*{\tt SUBROUTINE calc\_var(hpsi,psi1,sumvar2)}
\begin{tabular}{lcl}
 {\tt psi1(kdfull2,kstate)}& in & set of s.p. states for which
 variance of s.p. energies of calculated\\
 {\tt hpsi(kdfull2,kstate)} & in/out & array
 $H\rightarrow\psi_\alpha$, on input in $k$-space, on output in $r$-space
\\
 {\tt sumvar2} & out  & summed variance of s.p. energies\\
\end{tabular}
\\[4pt]
Computes the sum of variances of the s.p. energies,
$\langle\hat{\Delta h}^2|rangle$. 
\\
PGRcomm{The routine projects from each $hat{h}\psi_\alpha$
all s.p. states $\psi_\beta$ from the pool of states. That is too
much. The s.p. variance should be
$\sum_\alpha\langle|\psi_\alpha|(\hat{h}-\varepsilon_\alpha)^2|\psi_\alpha\rangle$
where $\varepsilon_\alpha=\langle|\psi_\alpha|\hat{h}|\psi_\alpha\rangle$.}




\subsubsection*{\tt SUBROUTINE forceTemp(amoy,occup,n,temp,mu)}
\begin{tabular}{lcl}
 {\tt amoy(1:n)} & in & given s.p. energies\\
 {\tt occup(1:n)} & in & given thermal occupation\\
 {\tt n} & in & number of s.p. states\\
 {\tt temp} & in & temperature\\
 {\tt mu} & out & emerging chemical potential\\
\end{tabular}
\\[4pt]
Determines chemical potential for given s.p. energies and temperature
by call to {\tt OccT1}.
\\
\PGRcomm{Obsolete and never used.}




\subsubsection*{\tt SUBROUTINE fermi\_init(ekmod,T,occup,ispinact)}
\begin{tabular}{lcl}
 {\tt ekmod(1:nstate)} & in & given s.p. energies\\
 {\tt T} & in & given temperature\\
 {\tt ispinact} & in & actual spin\\
 {\tt occup(1:kstate)} & in/out & initial occupation and resulting
 Fermi distribution for {\tt T}.\\
\end{tabular}
\\[4pt]
Determines Fermi distribution for given s.p. energies and
temperature. Searches appropriate chemical potential {\tt mu} by
bracketing. Use for repreated calls to {\tt FUNCTION occ}.
\\
\PGRcomm{This routine {\tt fermi\_init} and the related
{\tt FUNCTION occ} are never used, thus obsolete. May be removed.}



\subsubsection*{\tt SUBROUTINE srhomat(psi,aloc,psiorth,occuporth)}
\begin{tabular}{lcl}
 {\tt psi(1:kdfull2,1:kstate)} & in & set of s.p. wavefunctions, not orth-normalized\\
 {\tt psiorth(1:kdfull2,1:kstate)} & out & ortho-normalized natural orbitals\\
 {\tt aloc(1:2*kdfull2)} & in & actual local potential\\
 {\tt occuporth(1:kstate)} & out & occupation numbers for
 ortho-normalized states\\
\end{tabular}
\\[4pt] Computes the density matrix of initial state goiven by set of
wavefunctions {\tt psi} together with their occupations {\tt occup},
the latter communicated through module {\tt params}.  Then
diagonalizes the density matrix and computes on {\tt psiorth} the new
wavefunctions associated with diagonal representation of the density
matrix.  
\\
Finally updates running transformation matrix {\tt psitophi} which is
communicated and stored through module {\tt params}.
\\
\PGRcomm{Usage and propagation of  {\tt psitophi} is somewhat hidden
  because it is handled through a module. Needs to be explained somwhere.}


\subsubsection*{\tt SUBROUTINE scalar(tab1,tab2,scal,ispin, mess)}
\begin{tabular}{lcl}
 {\tt tab1(1:kdfull2,1:kstate)} & in & 1. set of s.p. wavefunctions\\
 {\tt tab2(1:kdfull2,1:kstate)} & in & 2. set of s.p. wavefunctions\\
 {\tt ispin(1:nstate)} & in & spin of s.p. states\\
 {\tt mess} & in & message for print inside routine\\
 {\tt scal(nstate,nstate)} & out & matrix of wavefunction overlaps\\
\end{tabular}
\\[4pt]


\subsubsection*{\tt SUBROUTINE cdiagspin(mat, eigen, vect, N)}
\begin{tabular}{lcl}
 {\tt mat(N,N)} & in & complex Hermitean matrix to be diagonalized\\
 {\tt N} & in & dimension of matrix\\
 {\tt eigen(N)} & out & resulting eigenbvalues\\
 {\tt Vect(N,N)} & out & resulting eigenstates\\
\end{tabular}
\\[4pt]
Driver routine for diagonalization of a complex Hermitean matrix of
dimension {\tt N} which consists in a two blocks for separate spin.
Refers for each single block to routine {\tt cdiag} and subsequent
library routines contained therein.


\subsubsection*{\tt SUBROUTINE indexx (n,arrin,indx)}
\begin{tabular}{lcl}
 {\tt n} & in & length of array\\
 {\tt arrin(1:n)} & in & array to be sorted\\
 {\tt indx(1:n)} & out & pointer array \\
\end{tabular}
\\[4pt]
Evaluates sorting of an array in ascending order.



\subsubsection*{\tt SUBROUTINE occupPerSpin(mess,Occ)}
\begin{tabular}{lcl}
 {\tt mess} & in & character variable with comment printed inside routine\\
 {\tt Occ(1:2)} & out & total number of particles in each spin\\
\end{tabular}
\\[4pt]
Computes number of particles in each sin block. Uses {\tt nstate} and
occupations {\tt occup} from module {\tt params}.



\subsubsection*{\tt CorrectEnergy2(Wref,Eref,w,E,Wout,nloc)}
\begin{tabular}{lcl}
 {\tt W(1:nloc)} & in & initial occupations numbers\\
 {\tt E(1:nloc)} & in & given s.p. energies\\
 {\tt Wref} & in & reference particle number to be reached\\
 {\tt Eref} & in &  reference sum of s.p. energies to be reached\\
 {\tt nloc} & in & actual number of states\\
 {\tt Wout(nloc)} & out  & readjusted occupation numbers\\
\end{tabular}
\\[4pt]
Final energy correction by one step along Fermi distribution
(using Taylor expansion about actual distribution),
see Eq.~(\ref{eq:})\PGRfoot{This equation from the
  theory part, yet to be written.}



\subsubsection*{\tt SUBROUTINE ordo\_per\_spin(psi)}
\begin{tabular}{lcl}
 {\tt psi(1:kdfull2,1:kstate)} & in/out & s.p. wavefunctions before
 and after reordering\\
\end{tabular}
\\[4pt]
Reorder states  in two blocks of spin up and down.
Applies that reshuffling to all relevant field of states,
s.p. wavefunctions {\tt psi}, spin per state {\tt ispin},
and occupations {\tt occup}. 
\\
\PGRcomm{Routine has been rendered obsolete by new initialization of
  states which produces immediately the correct sorting. But routine
  should be kept for possible later use (e.g., mixing states from
  different sources.}

\subsubsection*{\tt SUBROUTINE temperature(mu,T)}
\begin{tabular}{lcl}
 {\tt mu} & out & resulting chemical potential\\
 {\tt T} & out & resulting temperature\\
\end{tabular}
\\[4pt]
Takes s.p. energies {\tt amoy} and occupations {\tt occup} from module
{\tt params} and fits a Fermi distribution to it. Temperature and
chemical potentials of the fitted distribution are returned via list.
Calls a fitting routine {\tt lmdif1} using subroutine {\tt ff} as argument.

\subsubsection*{\tt SUBROUTINE ff(m,n,X,FVEC,IFLAG)}
\begin{tabular}{lcl}
 {\tt X(1:n)} & in & array handling chemical potential and temperature\\
 {\tt Fvec(1:m)} & out & array of mismatches of distributions\\
 {\tt n} & in & number of parameters of model, actually 2\\
 {\tt m} & in & number of entries in array\\
 {\tt iflag} & in & flaf possibly written (actually not used)\\
\end{tabular}
\\[4pt]
Mismatch of {\tt occup} (via modules {\tt params}) from Fermi
distribution to given chemical potential and temperature. To be used
in fitting routine {\tt lmdef1}.


\subsubsection*{\tt SUBROUTINE cproject(qin,qout,ispact,q0)}
\begin{tabular}{lcl}
 {\tt qin(1:kdfull2)} & in & s.p. wvaefunction to be projected\\
 {\tt q0(1:kdfull2,1:kstate)} & in & set of s.p. wavefunctions which
 is projected out from {\tt qin}\\
 {\tt ispact} & in & spin associated with {\tt qin}\\
 {\tt qout(1:kdfull2)} & out & projected s.p. wavefunction \\
\end{tabular}
\\[4pt] Projects away from {\tt qin} all contributions of the set 
{\tt q0}.  
\\ 
\PGRcomm{This routine may become obsolete if we recode the
  the variance in routine {\tt calc\_var} to meet the
  standard definition.}



\subsubsection*{\tt }
\begin{tabular}{lcl}
 {\tt } & &\\
\end{tabular}
\\[4pt]



\subsubsection*{\tt }
\begin{tabular}{lcl}
 {\tt } & &\\
\end{tabular}
\\[4pt]


\newpage


\section{Formula from Ann.Phys. paper}

\subsection{Mean-field propagation}
\label{sec:mf}

The starting point and dominant feature of the dynamics is the
propagation at the level of the mean field. In this paper, we are
dealing with the electron dynamics in metal clusters and we describe
it by time-dependent density functional theory at the level of the
Time-Dependent Local-Density Approximation (TDLDA) treated in the real
time domain \cite{Gro90,Gro96}.  It is augmented by a self-interaction
correction (SIC) approximated by average-density SIC (ADSIC)
\cite{Leg02} in order to attain correct ionization properties
\cite{Klu13} in the course of the dynamical simulation. TDLDA is
formulated within the usual Kohn-Sham picture in terms of a set of
occupied single-particle (s.p.) wavefunctions
$\{|\phi_\alpha\rangle,\alpha=1...N\}$. Their dynamics is described by
the time-dependent Kohn-Sham equation
\begin{equation}
  \I\partial_t|\phi_\alpha\rangle
  =
  \hat{h}[\varrho]|\phi_\alpha\rangle
\label{eq:KSwf}
\end{equation}
where $\hat{h}$ is the Kohn-Sham mean-field Hamiltonian which is a
functional of the instantaneous local density
$\varrho(\mathbf{r},t)=\sum_\alpha|\phi_\alpha(\mathbf{r},t)|^2$
\cite{Rei04aB,Dre90}. The time evolution delivered by
Eq. (\ref{eq:KSwf}) can be expressed formally by the
unitary one-body  time-evolution operator 
\begin{subequations}
\begin{equation}
  \hat{U}(t,t')
  =
  \hat{\mathcal{T}}\mathrm{ exp}\left(-i \int_t^{t'} \hat{h}(t'')dt''\right)
\label{eq:KSpropag}
\end{equation}
where $\hat{\mathcal{T}}$ is the time-ordering operator.
This yields a closed expression for the time-evolution of s.p. states
\begin{equation}
  |\phi_\alpha(t)\rangle
  =
  \hat{U}(t,t')|\phi_\alpha(t')\rangle.
\label{eq:KSpropag2}
\end{equation}
\end{subequations}

So far, TDLDA propagates pure states. Dissipation which we will add
later on leads inevitably to mixed states. This requires to generalize
the description from fully occupied s.p. wavefunctions to a one-body
density operator $\hat{\rho}$.  \PGR{Its representation in
  configuration space, i.e. in terms of a given set of s.p. states
  $|\varphi_i\rangle$, reads in general
  $\hat{\rho}=\sum_{ij}|\varphi_i\rangle\rho_{ij}\langle\varphi_j|$.
  By appropriate transformation of the s.p. basis, one can diagonalize
  the density matrix $\rho_{ij}$ which defines what are called natural
  orbitals. The natural orbitals representation of the one-body
  density operator then reads}
\begin{equation}
  \hat{\rho}
  =
  \sum_{\alpha=1}^\infty|\phi_\alpha\rangle W_\alpha\langle\phi_\alpha|
  \quad.
\label{eq:rhodiag}
\end{equation}
\PGR{The weights $W_\alpha$ represent} the probability with which a
state $|\phi_\alpha\rangle$ is occupied. The mean-field propagation
(\ref{eq:KSwf}) then becomes
\begin{equation}
  \I\partial_t\hat{\rho}
  =
  \left[\hat{h}[\varrho],\hat{\rho}\right]
\label{eq:KSrho}
\end{equation}
where $\hat{h}[\varrho]$ is formally the same as before and the
local density is now computed as
\begin{equation}
\varrho(\mathbf{r},t)=\sum_\alpha{W}_\alpha|\phi_\alpha(\mathbf{r},t)|^2 .
\label{eq:rhoW}
\end{equation}
The \PGR{(coherent)} pure mean-field propagation (\ref{eq:KSrho})
leaves the occupation weights $W_\alpha$ unchanged and propagates only
the s.p. states.  The mean-field propagation of an initial state
(\ref{eq:rhodiag}) then reads
\begin{eqnarray}
  \hat{\rho}(t)
  &=&
  \sum_{\alpha=1}^\infty
  |\phi_\alpha(t)\rangle W_\alpha\langle\phi_\alpha(t)|
%\nonumber\\  
%  &=&
  =
  \hat{U}(t,0)\hat{\rho}(0)\hat{U}^{-1}(t,0)
\label{eq:KSrhoevol}
\end{eqnarray}
where $\hat{U}$ is the mean-field evolution operator (\ref{eq:KSpropag}).

\subsection{RTA in quantum-mechanical framework}

The generalization of the one-body phase-space distribution
$f(\mathbf{r},\mathbf{p})$ to a quantum-mechanical mean-field theory
is the one-body density operator $\hat{\rho}$, or one-body density
matrix $\rho(\mathbf{r},\mathbf{r}')$ respectively. The equation of
motion for $\hat{\rho}$ including dynamical correlations reads in
general \cite{Rei85f,Goe86a}
\begin{eqnarray}
  \mathrm{i}\partial_t\hat{\rho}
  -
  \big[\hat{h},\hat{\rho}\big]
  &=&
  \hat{I}[\hat{\rho}]
  \quad.
\label{eq:EoMfull}
\end{eqnarray}
The left hand side embraces the mean-field propagation. It may be
time-dependent Hartree-Fock or the widely used LDA version of TDDFT.
The right-hand side consists of the quantum-mechanical collision term.
Motivated by the successful semi-classical RTA, we import
Eq.~(\ref{eq:VUUrelax}) for the quantum case as
%\begin{subequations}
\begin{eqnarray}
  \partial_t\hat{\rho}
%  &=&
%  -\mathrm{i}\big[\hat{h},\hat{\rho}\big]
%  -
%  \frac{1}{\tau_\mathrm{relax}}
%  \left(\hat{\rho}-\hat{\rho}_\mathrm{eq}[\hat{\rho}]\right) \nonumber \\
  &=&
  -\mathrm{i}\big[\hat{h},\hat{\rho}\big]
  -
  \frac{1}{\tau_\mathrm{relax}}
  \left(\hat{\rho}-\hat{\rho}_\mathrm{eq}[\varrho,\mathbf{j},E]\right)
  \;,
\label{eq:EoMbasic}
\end{eqnarray}
where $\hat{\rho}_\mathrm{eq}$ is the density operator of the thermal
equilibrium for local density $\varrho(\mathbf{r},t)$, current
distribution $\mathbf{j}(\mathbf{r},t)$ and total energy $E(t)$
given at that instant of time $t$.  These
constraining conditions are, in fact, functionals of the actual state
$\hat{\rho}$, i.e. $\varrho[\hat{\rho}]$, $\mathbf{j}[\hat{\rho}]$,
and $E[\hat{\rho}]$.  For the diagonal representation
Eq.(\ref{eq:rhodiag}) of the density operator $\hat{\rho}$, they read
%\begin{subequations}
\begin{eqnarray}
  \varrho(\mathbf{r})
  =
  \sum_\alpha \left|\phi_\alpha(\mathbf{r})\right|^2 W_\alpha
  &\;,\;&
%\\
  \mathbf{j}(\mathbf{r})
  =
  \sum_\alpha W_\alpha\phi_\alpha^*(\mathbf{r})
     \frac{\stackrel{\rightarrow}{\nabla}-\stackrel{\leftarrow}{\nabla}}
          {2\mathrm{i}}
     \phi_\alpha(\mathbf{r})
  \quad.
\end{eqnarray}
%\end{subequations}
The energy $E(t)$ is taken as the total energy because the
semi-classical concept of a local kinetic energy is
ambiguous in a quantum system.  This RTA equation (\ref{eq:EoMbasic})
looks innocent, but is very involved because many entries depend in
various ways on the actual state $\hat{\rho}(t)$. The self-consistent
mean field is a functional of the actual local density,
i.e. $\hat{h}=\hat{h}[\varrho]$. The instantaneous equilibrium density
$\hat{\rho}_\mathrm{eq}$ is the solution of the stationary, thermal
mean-field equations with constraint on the actual
$\varrho(\mathbf{r})$, $\mathbf{j}(\mathbf{r})$ and energy $E$, for
details see Appendix \ref{sec:hdenscurrE}.

The relaxation time $\tau_\mathrm{relax}$ is estimated in
semi-classical Fermi liquid theory, for details see appendix
\ref{sec:relaxtime}. For the metal clusters serving as test examples
in the following, it becomes
\begin{equation}
  \frac{\hbar}{\tau_\mathrm{relax}}
  =
  {0.40}\frac{\sigma_{ee}}{r_s^2}\frac{{E}^*_\mathrm{intr}}{N}
  \quad,
\label{eq:relaxtime}
\end{equation}
where $E^*_\mathrm{intr}$ is the intrinsic (thermal) energy of the
system (appendix \ref{app:eintr}), $N$ the actual number of particles,
$\sigma_{ee}$ the in-medium electron-electron cross section, and $r_s$
the effective Wigner-Seitz radius of the electron cloud.  \PGR{Note
  that $r_s$ is tuned to the average density of the electron cloud (appendix
  \ref{app:eintr}), because a spatially varying $\tau_\mathrm{relax}$
  would be very cumbersome to implement in a quantum mechanical
  context. This approximation is legitimate for metallic systems where
  the density remains generally close to the average.}


\subsection{Summary of the procedure}
\label{sec:summary}




\begin{figure}
%\setlength\unitlength{0.91mm}
\thicklines
\begin{center}
%\begin{sideways}
\fbox{\small%\footnotesize
%\begin{picture}(130,92)(0,108)
\begin{picture}(132,175)(-5,-175)
\put(0,-5){\mbox{
\begin{minipage}[t]{8cm}
\begin{flushleft}
$\;$starting point: 
\\[4pt]
  $\hat{\rho}(t_0)
  =
  \sum_\alpha|\phi_\alpha(t_0)\rangle W_\alpha(t_0)\langle\phi_\alpha(t_0)|$
\end{flushleft}
\end{minipage}
}}
\put(5,-19){\encircle{1}}
\put(3,-11.5){\vector(0,-1){25}}
\put(6,-30){\fbox{\tt tstep}}
\put(7,-25){\mbox{mean-field propagation: 
$%\begin{array}{l}
 |\phi_\alpha^\mathrm{(mf)}\rangle=\hat{U}(t_1,t_0)|\phi_\alpha(t)\rangle
\;,\;
%\\
 W_\alpha^\mathrm{(mf)}=W_\alpha(t)=\mbox{const.}
%\end{array}
$
}}
\put(0,-40){\mbox{
  $ \hat{\rho}_\mathrm{mf} =\hat{\rho}_\mathrm{mf}(t_1) 
  =
  \sum_\alpha|\phi_\alpha^\mathrm{(mf)}\rangle 
  W_\alpha^\mathrm{(mf)}\langle\phi_\alpha^\mathrm{(mf)}|$
}}
\put(5,-46){\encircle{2}}
\put(6,-54.5){\fbox{\tt srhomat}}
\put(3,-41.5){\vector(0,-1){15}}
\put(0,-50){\mbox{
\begin{minipage}[t]{8cm}
\begin{flushleft}
\hspace*{1.8em}express $\phi$ and $W$ through natural orbitals
\\[20pt]
  $\displaystyle
  \hat{\rho}_\mathrm{mf}(t_1)
  =
  \sum_\alpha|\phi_\alpha^\mathrm{nat}\rangle \tilde{W}_\alpha^\mathrm{nat}
  \langle\phi_\alpha^\mathrm{nat}|$
\end{flushleft}
\end{minipage}
}}
\put(44,-63){\vector(4,-1){24}}
\put(51,-62){\encircle{3}}
\put(61,-64){\fbox{\tt calcrhotot,calc\_current}}
\put(69,-70){\mbox{  $\varrho_\mathrm{mf}(\mathbf{r},t_1)\,,\,\mathbf{j}_\mathrm{mf}(\mathbf{r},t_1),E_\mathrm{mf}$}}

\put(87,-72){\vector(1,-1){12}}
%\put(117,-103.5){\vector(0,-1){8}}
\dottedline{1}(117,-103.5)(117,-110.5)
\put(117,-110.5){\vector(0,-1){1}}
\put(85,-77){\encircle{4}}
\put(97,-79){\fbox{\tt eqstate}}
\put(67,-87){\mbox{
\begin{minipage}[t]{8cm}
\begin{flushleft}
  density-constrained mean field (DCMF) %\\ $\hookrightarrow$ %$\Longrightarrow$
\\[3pt]
$\Rightarrow$ 1. 
$\begin{array}[t]{rcl}
  \hat{\rho}_{eq}
  &=&
  \hat{\rho}_{eq}[\varrho_\mathrm{mf}(\mathbf{r})\,,\,\mathbf{j}_\mathrm{mf}(\mathbf{r}),E_\mathrm{mf}]
\\[3pt]
  &=&
  \sum_\alpha|\phi'_\alpha\rangle  W'_\alpha\langle\phi'_\alpha|
\end{array}$
\\[4pt]
$\Rightarrow$ 2. intrinsic excitation energy $E^*_\mathrm{intr}$
\end{flushleft}
\end{minipage}
}}
%\put(79,-93.5){\vector(-4,-1){33}}
\dottedline{1}(77,-93.5)(47,-101.5)
\put(47,-101.5){\vector(-4,-1){1}}
\put(83,-110){\mbox{
\begin{minipage}[t]{8cm}
\begin{flushleft}
relaxation time:
\\[4pt]
$\hbar\tau_\mathrm{relax}^{-1}=0.40\,\sigma_{ee}r_s^{-2}\,E^*_\mathrm{intr}/N$
\end{flushleft}
\end{minipage}
}}
\dottedline{1}(83,-114)(31,-106)
\put(31,-106){\vector(-4,1){1}}
\put(4,177){\vector(0,-1){39}}
\put(0,-96){\mbox{
\begin{minipage}[t]{8cm}
\begin{flushleft}
\hspace*{1.8em}compose with rate $\tau_\mathrm{relax}$:
\\[4pt]
$\displaystyle
\hat{\rho}_\mathrm{mix}
= 
\hat{\rho}_\mathrm{mf} -
\frac{\Delta t}{\tau_\mathrm{relax}}\left[\hat{\rho}_\mathrm{mf}-\hat{\rho}_{eq}\right]
$
\end{flushleft}
\end{minipage}
}}
\put(3,-104){\vector(0,-1){31}}
\put(5,-121){\encircle{5}}
\put(6,-130.5){\fbox{\tt calcrhoeq}}
\put(0,-126){\mbox{
\begin{minipage}[t]{8cm}
\begin{flushleft}
\hspace*{1.8em}express new density through its natural orbitals:
\\[24pt]
  $\displaystyle
  \hat{\rho}_\mathrm{mix}
  =
  \sum_\alpha|\phi_\alpha(t_1)\rangle \tilde{W}_\alpha
  \langle\phi_\alpha(t_1)|$
\end{flushleft}
\end{minipage}
}}
\put(3,-140){\vector(0,-1){21}}
\put(5,-146.5){\encircle{6}}
\put(6,-155.5){\fbox{\tt temperature}}
\put(0,-151){\mbox{
\begin{minipage}[t]{8cm}
\begin{flushleft}
\hspace*{1.8em}final fine-tuning of $W_\alpha$ to reproduce $E_\mathrm{mf}$
\\[24pt]
  $\displaystyle
  \hat{\rho}(t_1)
  =
  \sum_\alpha|\phi_\alpha(t_1)\rangle W_\alpha(t_1)\langle\phi_\alpha(t_1)|$
\end{flushleft}
\end{minipage}
}}
\dottedline{1}(3,-166)(3,-171)(-1.5,-171)(-1.5,-4.2)(0,-4.2)
\put(0,-4.2){\vector(1,0){1}}
\end{picture}
}
%\end{sideways}
\end{center}
\caption{\label{fig:summary} Sketch of the scheme for performing one
  large time step $t_0\longrightarrow t_1=t_0\!+\!\Delta t$ in solving the RTA
  equations.  The numbers in open circles indicate the steps as
  outlined in the text.  }
\end{figure}




The solution of the RTA equations is rather involved. We explain the
necessary steps here from a practical side and unfold details in the
appendices. We briefly summarize the actual scheme for one step from
$t$ to $t\!+\!\Delta t$.  Note that mean-field propagation (actually
TDLDA) runs at a much faster pace than relaxation.  We resolve it by
standard techniques \cite{Cal00,Rei04aB} on a time step $\delta t$
which is much smaller (factor 10--100) than the RTA step $\Delta
t$. We summarize this TDLDA propagation in the evolution operator
$\hat{U}$ from Eq.~(\ref{eq:KSpropag}) and discuss only one RTA
step. Its sub-steps are sketched in Figure
\ref{fig:summary} and
explained in the following whereby the label here correspond to the
ones in the Figure:
\begin{enumerate}
   \item\label{it:TDLDA} We first propagate $\hat{\rho}$ by pure
     TDLDA.  This means that the s.p. states in representation
     (\ref{eq:rhodiag}) evolve as
     $|\phi_\alpha(t)\rangle\rightarrow
     |\phi_\alpha^\mathrm{(mf)}\rangle=\hat{U}(t+\Delta
     t,t)|\phi_\alpha(t)\rangle$, while the occupation weights
     $W_\alpha$ are kept frozen (pure mean-field propagation).
   \item\label{it:newrho} We compute density
     $\varrho(\mathbf{r},t+\Delta t)$, current
     $\mathbf{j}(\mathbf{r},t+\Delta t)$, and total energy
     $E_\mathrm{mf}$ associated to the TDLDA-propagated density matrix
     $\hat{\rho}_\mathrm{mf}$.
   \item\label{it:DCMF} We determine the thermal mean-field
     equilibrium state $\hat{\rho}_\mathrm{eq}$ constrained to the
     given $\varrho$, $\mathbf{j}$, and $E_\mathrm{mf}$.  This is
     achieved by Density-Constrained Mean Field (DCMF) iterations 
     as outlined in
     Appendix \ref{sec:hdenscurrE}.  The actual equilibrium state
     $\hat{\rho}_\mathrm{eq}$ is represented by new s.p. states
     $\{|\phi'_{\alpha}\rangle\}$ and new occupation numbers $W'_\alpha$
     in diagonal form (\ref{eq:rhodiag}).
   \item \label{it:compo} We compose the new density matrix from the
     TDLDA propagated state $\hat{\rho}_\mathrm{mf}$ and the
     equilibration driving term
     $\hat{\rho}_\mathrm{mf}-\hat{\rho}_\mathrm{eq}$ with the
     appropriate weight $\Delta t/\tau_\mathrm{relax}$, as outlined in
     Appendix \ref{sec:mix}.  The relaxation time
     Eq. (\ref{eq:relaxtime}) requires the actual intrinsic excitation
     energy $E^*_\mathrm{intr}$ which is also obtained from DCMF, see
     appendix \ref{app:eintr}.
   \item \label{it:natural} 
     We diagonalize the state emerging from
     step \ref{it:compo} to natural-orbital representation
     Eq. (\ref{eq:rhodiag}).  This yields the s.p. states
     $\{|\phi_\alpha(t\!+\!\Delta{t})\rangle\}$ for the next step and
     preliminary new occupations $\tilde{W}_\alpha$.
   \item \label{it:therm} 
     After all these steps, the initial energy
     $E_\mathrm{mf}=E_\mathrm{TDLDA}(t)$ may not be exactly
     reproduced. We may remain with a small energy mismatch as
     compared to the goal $E_\mathrm{mf}$.  We now apply a small
     iterative thermalization step to readjust the energy, as outlined
     in Appendix \ref{sec:corriter}. This then yields the final
     occupation weights $W_\alpha(t\!+\!\Delta{t})$ which comply with
     energy conservation.
\end{enumerate}
The scheme can be used also in connection with absorbing boundary
conditions \cite{Cal00,Rei06c}. The particle loss will be mapped
automatically to loss of occupation weights in step \ref{it:compo}. A
word is in order about the choice of the time steps. The $\delta t$ for
propagation of TDLDA is limited by the maximal energy on the grid
representation and thus very small (for Na clusters typically 0.005
fs). The stepping for the relaxation term needs only to resolve the
changes in the actual mean field which is achieved already with
$\Delta t\approx 0.5$ fs. We have tested a sequence of $\Delta t$ and
find the same results for all $\Delta t\leq 0.5$ fs. Changes appear
slowly above that value.  For reasons of efficiency, we thus use the
largest safe value of $\Delta t= 0.5$ fs.


\PGR{A word is in order about the range of applicability of the RTA
  for finite fermion systems. The relaxation time
  $\tau_\mathrm{relax}$ is allowed to depend on time which allows to
  accomodate changes of the dynamical state. But $\tau_\mathrm{relax}$
  is at each instant if time one global number chosen according to the
  average electron density.  This requires systems with only small
  density variations in the bulk as it holds typically for metallic
  bonds.  The RTA is insensitive to many details of the VUU collision
  term as energy- and angle-dependent scattering cross sections or a
  broad spectrum of relaxation rates. However, these details are
  usually resolved only (if at all) for fast and energetic processes
  which are anyway deep in the regime of semi-classical VUU. The grossly
  averaged treatment of RTA is acceptable for not too fast and not too
  energetic processes in compact metallic systems.}

\subsection{Numerical representation and computation of relevant observables  }
\label{sec:observ}

The numerical implementation of TDLDA is done in standard
manner~\cite{Cal00,Rei04aB}.  The coupling to the ions is mediated by
soft local pseudopotentials~\cite{Kue99}.  The Kohn-Sham potential is
handled in the Cylindrically Averaged Pseudo-potential Scheme (CAPS)
\cite{Mon94a,Mon95a}, which has proven to be an efficient and reliable
approximation for metal clusters close to axial symmetry.
Wavefunctions and fields are thus represented on a 2D cylindrical grid
in coordinate space \cite{Dav81a}.  For the typical example of the
Na$_{40}$ cluster, the numerical box extends up to 104 a$_0$ in radial
direction and 208 a$_0$ along the $z$-axis, while the grid spacing is
0.8 a$_0$. To solve the (time-dependent) Kohn-Sham equations
(\ref{eq:KSwf}) we use time-splitting for time
propagation~\cite{Fei82} and accelerated gradient iterations for the
stationary solution \cite{Blu92}. The Coulomb field is computed with
successive over-relaxation \cite{Dav81a}.  We use absorbing boundary
conditions~\cite{Cal00,Rei06c}, which gently absorb all outgoing
electron flow reaching the bounds of the grid and thus prevent
artifacts from reflection back into the reaction zone.  We take the
exchange-correlation energy functional from Perdew and
Wang~\cite{Per92}.

A great manifold of observables can be deduced from the
$\hat{\rho}(t)$ thus obtained. We will consider in the following the
dipole signal, dipole spectrum, ionization, angular distribution of
emitted electrons, and entropy. We focus here on the dipole moment
along symmetry axis $z$, which is obtained from the local density as
$\langle\hat{d}_z\rangle(t)=\int{d}^3r\,d_z(\mathrm{z})\varrho(\mathrm{r})$
where $d_z(\mathrm{z})=z$ is the (local) dipole operator. The dipole
strength distribution is computed with the methods of spectral
analysis \cite{Cal97b}. It is attained by an instantaneous
dipole-boost excitation, collecting $\langle\hat{d}_z\rangle(t)$
during propagation, and finally Fourier transforming
$\langle\hat{d}_z\rangle(t)$ into frequency domain. The angular
  distribution of emitted electrons is obtained from recording the
absorbed electrons as in TDLDA \cite{Poh04b,Rei06aR}. The angular
distribution is characterized by the anisotropy parameter
$\beta_2$, the leading parameter in the photo-electron angular cross
section $d\sigma/d\Omega \propto (1+\beta_2 P_2(cos(\theta)+....)$
\cite{Wop10a,Wop10b} where $P_2$ is the second order Legendre
polynomial and $\theta$ the direction with respect to laser
polarization axis (here $z$-axis in 2D cylindrical geometry). A
specific quantity to track relaxation processes is the one-body
entropy which is computed in diagonal representation
(\ref{eq:rhodiag}) by the standard expression \cite{Rei98aB}
\begin{equation}
  S
  =
  - \sum_\alpha\left[
    W_\alpha\log W_\alpha
    +
    (1\!-\!W_\alpha)\log (1\!-\!W_\alpha)
  \right]
\label{eq:entropy}
\end{equation}
in units of Boltzmann constant. 
It serves as a direct indicator of thermalization and allows to 
read off the typical time scale of relaxation processes. 




\end{document}
