%%%%%%%%%%%%%%%%%%%%%%%%%%%%%%%%%%%%%%%%%%%%%%%%%%%%%%%%%%
%%% PLEASE COMPILE WITH XELATEX TO GET THE FONTS RIGHT %%%
%%%%%%%%%%%%%%%%%%%%%%%%%%%%%%%%%%%%%%%%%%%%%%%%%%%%%%%%%%
\usepackage[xetex]{geometry}
\usepackage[utf8]{inputenc}
\usepackage{xcolor}
\usepackage{parskip}
\usepackage{mathtools}
\usepackage{physics}
\usepackage[xetex]{hyperref}
\usepackage{hypertoc}
\usepackage{enumitem}
\usepackage{fancyvrb}
\usepackage{mathspec}
\usepackage{titlesec,titletoc}
\usepackage[singlelinecheck=false]{caption}
\usepackage{tikz}

\geometry{top=1in,bottom=1in,left=1in,right=1.375in}
\graphicspath{{User_manual.figures/}}

%\setmathfont(Latin){Minion Pro}
%\setmathrm{Minion Pro}
%\setmathsf{Myriad Pro Light} %% Make this match the headline font
%\exchangeforms{phi}			 %%   to get math in the headings right
%\defaultfontfeatures{
%	Ligatures={Required,Common,Contextual,Rare,Historic,TeX},
%	Style=Swash,
%	Contextuals={Swash,WordInitial,WordFinal}
%}
%\setmainfont{Minion Pro}
%[
%	UprightFeatures={
%		SizeFeatures={
%			{Size={-8.4},Font=* Caption},
%			{Size={8.4-13},Font=*},
%			{Size={13-19.9},Font=* Subhead},
%			{Size={19.9-},Font=* Display}
%		},
%	},
%	BoldFeatures={
%		SizeFeatures={ 
%			{Size={-8.4},Font=* Bold Caption},
%			{Size={8.4-13},Font=* Bold},
%			{Size={13-19.9},Font=* Bold Subhead},
%			{Size={19.9-},Font=* Bold Display}
%		},
%	},
%	ItalicFeatures={
%		SizeFeatures={ 
%			{Size={-8.4},Font=* Italic Caption},
%			{Size={8.4-13},Font=* Italic},
%			{Size={13-19.9},Font=* Italic Subhead},
%			{Size={19.9-},Font=* Italic Display}
%		},
%	},
%	BoldItalicFeatures={
%		SizeFeatures={ 
%			{Size={-8.4},Font=* Bold Italic Caption},
%			{Size={8.4-13},Font=* Bold Italic},
%			{Size={13-19.9},Font=* Bold Italic Subhead},
%			{Size={19.9-},Font=* Bold Italic Display}
%		},
%	},
%]
%\setsansfont{Myriad Pro}
%\newfontfamily\headlinefont{Myriad Pro Light}
\newfontfamily\headlinefont{Latin Modern Sans}

%\linespread{1.3} % Only touch if the linespacing is insufficient

\definecolor{niceBlue2}{RGB}{16,117,175}
\colorlet{activeColor}{niceBlue2}
\hypersetup{
	colorlinks=true,
	linkcolor=activeColor,
	urlcolor=activeColor,
	citecolor=activeColor,
}\urlstyle{same}

\titleformat{\section}[hang]
	{\headlinefont\LARGE}
	{\thesection}
	{12pt}
	{\color{activeColor}}
\titlespacing{\section}{0em}{4ex}{1ex}
\titleformat{\subsection}[hang]
	{\headlinefont\large}
	{\thesubsection}
	{12pt}
	{\color{activeColor}}
\titlespacing{\subsection}{0em}{4ex}{1ex}
\titleformat{\subsubsection}[hang]
	{\headlinefont\itshape}
	{\thesubsubsection}
	{12pt}
	{\color{activeColor}}
\titlespacing{\subsubsection}{0em}{4ex}{1ex}

\titlecontents{section}[0em]
	{\addvspace{2ex}\headlinefont\large}
	{\hspace*{1.5em}{\contentslabel{1.5em}}}
	{\hspace*{1.5em}{\contentslabel{1.5em}}}
	{\titlerule*[0.7em]{.}\contentspage}
	[]
\titlecontents{subsection}[2.025em]
	{\addvspace{1ex}\headlinefont\normalsize}
	{\hspace*{2em}{\contentslabel{2em}}\color{activeColor}}
	{\hspace*{2em}{\contentslabel{2em}}\color{activeColor}}
	{\titlerule*[0.7em]{.}\contentspage}
	[]
\titlecontents{subsubsection}[4.24em]
	{\headlinefont\itshape\small}
	{\hspace*{3em}{\contentslabel{3em}}}
	{\hspace*{3em}{\contentslabel{3em}}}
	{\titlerule*[0.7em]{.}\contentspage}
	[]

\captionsetup[table]{labelfont={sc,color=activeColor},name=Table}
\captionsetup[figure]{labelfont={sc,color=activeColor},name=Figure}

\renewcommand{\texttt}[1]{{\bfseries\ttfamily #1}}
\usetikzlibrary{shapes}
\newcommand*\circled[1]{
	\tikz[baseline=(char.base)]{
		\node[shape=circle, draw, inner sep=2pt] (char) {#1};
	}
}
\makeatletter
	\newcommand{\showfontsize}{\f@size{} pt}
\makeatother
\newcommand{\cmnt}[1]{\color{red}\uppercase{#1}}
\newcommand{\eq}[1]{Equation~(\ref{#1})}	   % \eq{label}  --> Equation (3)
\newcommand{\scn}[1]{Section~\ref{#1}}	   % \scn{label} --> Section 1.2
\newcommand{\fig}[1]{Figure~\ref{#1}}	   % \fig{label} --> Figure 1.3
\newcommand{\figs}[1]{Figures~\ref{#1}}	   % \fig{label} --> Figure 1.3
\newcommand{\rf}[1]{~Ref.$\!$\citen{#1}}	   % \rf{label}  --> ~Ref. [11]
\newcommand{\rfs}[1]{~Refs.$\!$\citen{#1}} % \rfs{label} --> ~Refs. [11,15,21-23]
\newcommand{\tab}[1]{Table~\ref{#1}}		   % \tab{label} --> Table 2.4
\newcommand{\tabs}[1]{Tables~\ref{#1}}	   % \tabs{label},~{label} --> Tables 2.4, 1.3

\renewcommand{\topfraction}{.85}
\renewcommand{\bottomfraction}{.7}
\renewcommand{\textfraction}{.15}
\renewcommand{\floatpagefraction}{.66}
\renewcommand{\dbltopfraction}{.66}
\renewcommand{\dblfloatpagefraction}{.66}