\documentclass[12pt]{article}
\usepackage[latin2]{inputenc}
%\usepackage{german}
\usepackage{epsfig}
\usepackage{amsmath}
\usepackage{graphicx}
\usepackage{hyperref} 
\usepackage{xcolor}
\definecolor{dark}{rgb}{0.4,0.15,0.15}
\definecolor{dark-blue}{rgb}{0.15,0.15,0.4}
\definecolor{medium-blue}{rgb}{0,0,0.5}
\hypersetup{
    colorlinks, linkcolor={black},
    citecolor={dark-blue}, urlcolor={medium-blue}
}

\textwidth 15.8cm
\textheight 23cm
\topmargin -2.5cm
\oddsidemargin 0.0cm
\evensidemargin 0.0cm
\pagestyle{empty}
%\partopsep -33pt
\parsep 12pt
%\topskip -33pt
\parskip 5pt
\parindent 0pt

\renewcommand{\baselinestretch}{1.01}
\newcommand{\onehalf}{\frac{1}{2}}
\renewcommand{\vec}[1]{\mathbf #1}

\unitlength 1mm


\begin{document}

\title{Namelist input parameters for the cluster 3D code}
\author{The Erlangen--Toulouse collaboration}
\date{Started 26. February 2019}
\maketitle


These notes serve to explain the input structure of the
cluster/molecule 3D code. It explains the full capabilities of the
code. We want to publish only a subset.  All options addressed in
namelist {\tt EXTENSIONS}, {\tt FSIC2}, and {\tt SURFACE} should not
appear in the public version. The switches to these parts are immersed
in the code with preproccesor variables {\tt extended}, {\tt fsic},
and {\tt raregas}. We should keep the code in the repository always in
its complete version and write a little piece of software which reads
the source files and produces new sources without the line enclosed be
the {\tt\#if(extended)}, {\tt\#if(extended)}, and {\tt\#if(raregas)}
closures.
\\
Note also the old input files have to changed. The namelist
``\&EXTENSIONS'' has to follow namelist ``\&DYNAMIC'' and all non-public
input parameters have to be moved to there. Moreover,
The namelist ``\&FSIC'' has to renamed ``\&FSIC2''.
A few examples with input in old style are collected in the directory
{\tt examples} while examples with input in new style com in
{\tt samples}.

\bigskip


The input is arranged in the following three input files:
\\[-28pt]
\begin{center}
\begin{tabular}{ll}
\hline
 {\tt for005} & defines the qualifier {\tt <name>} for 
                the other {\tt for005...} files\\
 {\tt for005.<name>} & general input for settings, static and
 dynamics\\
 {\tt for005ion.<name>} & ionic configuration of cluster\\
 {\tt for005surf.<name>} & atomic configuration of substrate (optional)\\
\hline
\end{tabular}
\end{center}


The input for {\tt for005.<name>} is sorted into Fortran namelists.
The following tables list all input variables entered through these
namelists and gives a short explanation. Comments and issues for
development are marked in red. A mere \textcolor{red}{???} indicates a
variable which should be considered to be shifted to namelist {\tt extension}.
\\
\begin{tabular}{ll}
\hline
\multicolumn{2}{c}{Namelist {\tt GLOBAL}} in {\tt for005.<name>} \\
\hline
\multicolumn{2}{c}{\it choice of system} \\
\hline
{\tt kxbox            }& nr. of grid points in $x$ direction\\
{\tt kybox            }& nr. of grid points in $y$ direction\\
{\tt kzbox            }& nr. of grid points in $z$ direction\\
& box sizes must fulfill {\tt kxbox}$\geq${\tt kybox}$\geq$ {\tt kzbox}\\
{\tt dx,dy,dz,        }& grid spacing (in  Bohr) for the 3D numerical grid - if negative this will \\
&be set to an optimal value and a value will be suggested for KXBOX \\
&in file NX - the code stops and has to be restarted\\
&the grid size is defined before compilation in {\tt params.F90}\\
&it has to be correlant with pseudopotentails\\
&corresponds to ecut in solid state\\
{\tt tcoulfalr} & switch to FALR Coulomb solver (else exact solver)  \\
{\tt iswitch\_interpol} &  switch to interpolated grid for PsP \\
{\tt numspin}          & number of spin components (2=full spin treatment)\\
                       & (1=spin averaged, possible problem for ADSIC)
\textcolor{red}{???}\\
{\tt kstate           }& maximum nr. of s.p. states which is possible (greater than {\tt nclust})\\
{\tt nclust           }& number of QM electrons \\
\\
{\tt nion2            }& selects type of ionic background \\
                       &  0 $\rightarrow$ jellium background \\
                       &  1 $\rightarrow$ background from ionic pseudo-potentials\\
                       &  2 $\rightarrow$ background read in from {\tt potion.dat}\\
{\tt temp} &  electronic temperature in static iteration \\
{\tt nion             }& number of cluster ions\\
{\tt nspdw            }& number of spin down electrons \\
{\tt radjel           }& Wigner-Seitz radius of jellium background\\
{\tt surjel         }& surface thickness of jellium background\\
{\tt bbeta         }& quadrupole deformation of jellium background\\
{\tt gamma         }& triaxiality of jellium background\\
{\tt itback} &  nr. of iterations for jellium background \\
{\tt rotclustx,y,z } & vector fo angle of initial rotation of ions\\
{\tt scaleclustx} & scaling of ionic configuration along x-axis  \\
{\tt scaleclusty} & scaling of ionic configuration along y-axis   \\
{\tt scaleclustz} & scaling of ionic configuration along z-axis   \\
{\tt scaleclust} & if $\neq 1$: scaling of ionic configuration along x-y-z  \\
{\tt shiftclustx} & shift of ionic configuration along x-axis  \\
{\tt shiftclusty} & shift of ionic configuration along y-axis  \\
{\tt shiftclustz} & shift of ionic configuration along z-axis  \\
\hline
\end{tabular}

\begin{tabular}{ll}
\hline
\multicolumn{2}{c}{Namelist {\tt GLOBAL}} in {\tt for005.<name>} \\
\hline
\multicolumn{2}{c}{\it initialization of wave functions} \\
\hline
{\tt b2occ            }& deformation for initial harmonic oscillator wf's\\
{\tt gamocc           }& triaxiality for initial harmonic oscillator wf's\\
{\tt osfac} & factor on initial oscillator radius  \\
{\tt deocc            }& nr. of states above Fermi energy (determines nr. of
states)\\
{\tt shiftWFx         }& shift of initial wavefunctions in x direction \\
{\tt ishiftCMtoOrigin }& switch to shift center of mass of cluster to
origin of box\\
{\tt ispinsep         }& initialize wavefunctiosn with some spin asymmetry\\
{\tt init\_lcao       }& choice of basis for wavefunction initialization \\
& =0 $\Longrightarrow$ harmonic oscillator functions (center can be
moved
  by {\tt shiftWFx})
\\
& =1 $\Longrightarrow$ atomic orbitals = WFs centered at ionic sites
\\
\hline
\multicolumn{2}{c}{\it convergence issues} \\
\hline
{\tt e0dmp            }& damping parameter for static solution of Kohn-Sham equations\\
& (typically about the energy of the lowest bound state)\\
{\tt epswf            }& step size for static solution of Kohn-Shahm
equations (of order of 0.5)\\
{\tt epsoro           }& required variance to terminate static iteration (order of 10$^{-5}$)\\
{\tt occmix} &  mixing factor: new (thermal) occupation to old \\
{\tt endcon} &  requires precision variance, termination criterion \\
\hline
\end{tabular}

\begin{tabular}{ll}
\hline
\multicolumn{2}{c}{Namelist {\tt GLOBAL}} in {\tt for005.<name>} \\
\hline
\multicolumn{2}{c}{\it yet to be sorted} \\
\hline
{\tt dpolx} & add stationary dipole field in x-direction
\textcolor{red}{(better placed in {\tt dynamic}?)}\\
{\tt dpoly} & add stationary dipole field in y-direction  \textcolor{red}{(better placed in {\tt dynamic}?)}  \\
{\tt dpolz} & add stationary dipole field in z-direction  \textcolor{red}{(better placed in {\tt dynamic}?)}  \\
\hline
\end{tabular}

\newpage

\begin{tabular}{ll}
\hline
\multicolumn{2}{c}{Namelist {\tt DYNAMIC}} in {\tt for005.<name>} \\
\hline
\multicolumn{2}{c}{\it numerical and physical parameters for statics and dynamics} \\
\hline
{\tt dt1              }& time step for propagating electronic wavefunctions,  $\frac{\Delta t}{\Delta x^{2}}\leq 1$\\
{\tt ismax            }& maximum number of static iterations \\
{\tt idyniter         }& switch to s.p. energy as E0DMP for 'iter$>$idyniter'\\
{\tt ifhamdiag} & diagonalization of m.f. Hamiltonian in static step\\
& (presently limited to fully occupied configurations)\\
{\tt itmax            }& number of time steps for electronic propagation\\
{\tt ifexpevol} & exponential evolution 4. order instead of TV splitting\\
{\tt iffastpropag} & accelerated time step in TV splitting\\
  & (for pure electron dynamics, interplay with absorbing b.c. ??)\\
{\tt irest            }& switch to restart dynamics from file 'save'\\
{\tt istat            }& switch to read wavefunctions from file 'rsave'\\
{\tt } &\hspace*{1em}it continues static iteration for 'ismax$>$0' \\
{\tt } &\hspace*{1em}it starts dynamics from these wf's for 'ismax=0' \\
{\tt idenfunc} & choice of density functional for LDA\\
               & 1 $\rightarrow$ Perdew \& Wang 1992 (default setting)\\
               & 2 $\rightarrow$ Gunnarson \& Lundquist\\
               & 3  $\rightarrow$ only exchange in  LDA \\
{\tt isave            }& saves results after every 'isave' steps \\
{\tt                  }& on file 'rsave' in and after static iteration\\
{\tt                  }& on file 'save' in dynamic propagation\\
{\tt ipseudo          }& switch for using pseudo-densities to represent substrate\\
{\tt                  }& atoms \\
{\tt ipsptype         }& type of pseudopotentials: 0 = soft local (errf);\\
                       & 1 = full Goedecker; 2 = local Goedecker;\\
                       & 3 = read from file goed.asci (no need to specify)  ;\\
                       & 4 = semicore read from file goed.asci\\
{\tt directenergy}   & {\tt .true.} = direct computation of energy \\
                       & (only for {\tt LDA}, {\tt Slater}, {\tt KLI})\\
{\tt ifsicp           }& selects type of self-interaction correction\\
    &  0 = pure LDA, 1 = SIC-GAM, 2 = ADSIC; 3 = SIC-Slater; \\
    &  4 = SIC-KLI; 5 = exact exchange. \\
    & 7 = localized SIC, 8 = full SIC (real), 9 = full SIC(complex)
     in extended branch {\tt fsic}\\
{\tt icooltyp         }& type of cooling (0=none, 1=pseudo-dynamics,\\
{\tt                  }& 2=steepest descent, 3=Monte Carlo)\\
{\tt ifredmas         }& switch to use reduced mass for ions in dynamics\\
{\tt ionmdtyp         }& ionic propagation
                         (0=none, 1=leap-frog, 2=velocity Verlet)\\
 & \textcolor{red}{(Should we move ``leap-frog'' to the extended version???)}\\
{\tt ntref}& nr. time step after which absorbing bounds are deactivated
\\
{\tt nabsorb}          & number of absorbing points on boundary (0 switches off) 
\\
{\tt powabso}          & power of absorbing boundary conditions
\\
{\tt ispherabso}       & switch to spherical mask in absorbing bounds
\\
\hline
\end{tabular}

\begin{tabular}{ll}
\hline
\multicolumn{2}{c}{Namelist {\tt DYNAMIC}} in {\tt for005.<name>} \\
\hline
\multicolumn{2}{c}{\it way of excitation} \\
\hline
{\tt centfx           }& initial boost of electronic wavefuncftions in x-direction\\
{\tt centfy           }& initial boost of electronic wavefuncftions in y-direction\\
{\tt centfz           }& initial boost of electronic wavefuncftions in z-direction\\
{\tt shiftinix} & initial x-shift of electronic wavefunctions \\
{\tt shiftiniy} & initial y-shift of electronic wavefunctions  \\
{\tt shiftiniz} & initial z-shift of electronic wavefunctions  \\
{\tt tempion          }& initial temperature of cluster ions \\
{\tt ekmat} & initial kinetic energy of substrate atom (boost in $x$, in eV)\\
{\tt itft   }& choice of shape of laser pulse \\
&   1 = ramp laser pulse, sine switching on/off\\
&   2 = gaussian laser pulse \\
&   3 = cos$^2$ pulse\\
{\tt tnode  }& time (in fs) at which pulse computation starts\\
{\tt deltat }& length of ramp pulse ({\tt itft = 1}), in fs\\
{\tt tpeak  }& time (in fs, relative to {\tt tnode}) at which peak is reached\\
& (for {\tt itft} = 1 and 2, pulse length becomes 2*{\tt tpeak})\\
{\tt omega  }& laser frequency (in Ry)\\
{\tt e0     }& laser field strength in Ry/Bohr\\
{\tt e1x,e1y,e1z   }& orientation of pulse\\
{\tt e0\_2  }& field strength of second laser pulse (only {\tt itft=3}) \\
{\tt phase2 }& phase of second pulse \\
{\tt omega2 }& frequency of second pulse\\
{\tt tstart2}& initial ime of second pulse\\
{\tt tpeak2} & peak time of 2. pulse (pulse length is {\tt 2*tpeak2})\\
{\tt iexcit} & modus of excitation (0=shifts, 1=rotation)\\
{\tt iangmo} & switch to compute angular momentum\\
{\tt irotat} & axis of rotation for excitation (x=1,y=2,z=2,xyz=4)\\
{\tt phirot} & angle of rotation for excitation (in units of degree)\\
{\tt nhstate,npstate}& nr. of hole and particle state for $1ph$
                      excitation\\
                     & this option can only be run from {\tt istat=1}
\\
{\tt phangle}        & angle of ``rotation'' into a $1ph$ state\\
{\tt phphase}        & phase of ``rotation'' into a $1ph$ state\\
& {\tt phangle} and {\tt phphase} must be different, ideally
 {\tt phangle}-{\tt phphase}=90\\
\hline
\end{tabular}

\begin{tabular}{ll}
\hline
\multicolumn{2}{c}{Namelist {\tt DYNAMIC}} in {\tt for005.<name>} \\
\hline
\multicolumn{2}{c}{\it flags for observables} \\
\hline
{\tt iplotorbitals} & switch to print plot-file {\tt pOrbitals} for
all static states \\
{\tt iemomsRel        }& multipole momentes of electron density \\
{\tt                  }& relative to origin (0) or c.m. of cluster (1)\\
{\tt istinf           }& modulus for printing information in static iteration \\
{\tt ifspemoms        }& switch to compute and print spatial s.p. moments\\
{\tt iftransme        }& switch to compute and print transition m.elements\\
{\tt irhoint\_time   }& modulus for printing slices of integrated densities\\
{\tt jstinf           }& modulus for printing information in dynamic \\
{\tt jinfo            }& modulus for printing dynamical information on {\tt
  infosp.<name>} \\
{\tt jdip             }& modulus for printing dipole moments on {\tt pdip.<name>}\\
{\tt jquad            }& modulus for printing quadrupole moments on {\tt pquad.<name>}\\
{\tt jesc            }& modulus for printing ionization {\tt pescel.<name>}\\
{\tt jdiporb} & modulus for printing dipoles for s.p. states on {\tt pdiporb.xyz} \\
{\tt jenergy          }& modulus for printing energy information on {\tt penergies.<name>} \\
{\tt jener} &  modulus for printing ionic energies\\
{\tt iflocaliz}        & activates computation of Becke's localization
\\
{\tt jelf}             & modulus for anaylzing and printing electron
localization in dynamics
\\
 & various files are written of the form {\tt pelf*.<name>}
\\
{\tt iflocaliz}             & modulus for anaylzing and printing electron
localization in statics
\\
{\tt jstinf}           & modulus for printing s.p. energies and variances
\\
{\tt jpos}           & modulus for printing ionic positions on {\tt pposion.<name>}
\\
{\tt jvel}           & modulus for printing ionic velocities on {\tt pvelion.<name>}
\\
{\tt jstateoverlap}  & switch to compute overlap of static state
                       with\\
                     & the state directly after dynamical
                       initialization
\\
{\tt jforce} & modulus for printing ionic forces \\
{\tt jgeomion} &  modulus for printing global measures of ionic configuration \\
{\tt jang} & modulus for printing electronic angular momentum  \\
{\tt jangabso} &  modulus for printing angular distribution of emitted
electrons\\
{\tt jspdp} &  modulus for printing spin dipole momenta \\
{\tt jposcm} &  modulus for printing electronic center of mass \\
{\tt ipasinf} &  modulus for printing general information along dynamics\\
{\tt jgeomel} &   modulus for printing global measures of electronic geometry \\
{\tt jelf} &  modulus for evaluating an printing electron localization\\
{\tt jmp} & modulus for storing information for PES \\
{\tt jnorms} & modulus for printing s.p. norms and ionization probabilities \\
{\tt jcharges} & modulus for printing radially averaged charge distribution \\
{\tt drcharges} & radial distance for scanning radially averaged charge distribution \\
{\tt jplotdensitydiff} &  modulus for printing $\rho(t)-\rho(0)$ along
x-axis\\
{\tt jplotdensitydiff2d} &   modulus for printing 2D cuts of $\rho(t)-\rho(0)$ \\
{\tt jplotdensity2d} &  modulus for printing 2D cuts of density  \\
\hline
\end{tabular}

\begin{tabular}{ll}
\hline
\multicolumn{2}{c}{Namelist {\tt DYNAMIC}} in {\tt for005.<name>} \\
\hline
\multicolumn{2}{c}{\it excitation and observables mixed} \\
\hline
{\tt phi} & phase of laser pulse, inactive for electrons, strange
for ions \textcolor{red}{???} \\
{\tt projcharge} &  charge of ionic projectile \\
{\tt projvelx} &  x-velocity of ionic point-charge  projectile\\
{\tt projvely} &  y-velocity of ionic point-charge  projectile  \\
{\tt projvelz} &  z-velocity of ionic point-charge  projectile  \\
{\tt projinix} &  initial x-coordinate of ionic point-charge projectile  \\
{\tt projiniy} &  initial y-coordinate of ionic point-charge  projectile  \\
{\tt projiniz} &  initial z-coordinate of ionic point-charge  projectile  \\
{\tt modionstep} & modulus for ion step (nr. of electron steps per ion
step)\\
{\tt ispidi} & =1  switches to initialization by spin-dipole boost\\
{\tt izforcecorr} & =1 enforce zero-force condition, =0 tests
condition, =-1 disables all  \\
{\tt dinmargin} & margin defining inner box in connection with
Gaussian pseudo-densities \textcolor{red}{check default}\\
{\tt iangabso} & option for origin for angular distribution (1=box,
2=c.m.) \\
{\tt ipes} & activates preparation of measuring points for PES \\
{\tt nangtheta} & number of PAD angular cones in $\theta$ direction \\
{\tt nangphi} &  number of PAD angular cones in $\phi$ direction \\
{\tt delomega} &  space angle of angular cones in PES \textcolor{red}{check}\\
{\tt angthetal} & lower angle  $\theta$ for PES evaluation \\
{\tt angthetah} & upper angle  $\theta$ for PES evaluation  \\
{\tt angphil} &  lower angle  $\phi$ for PES evaluation \\
{\tt angphih} &  upper angle  $\phi$ for PES evaluation \\
{\tt ifreezekspot} &  =1 freezes KS potential at stage of time=0\\
{\tt ifixcmion} & switch to fix c.m. during ionic motion \\
{\tt ekin0pp} & kinetic energy for initial boost of electrons
\underline{and} ions \\
{\tt vxn0} & boost velocity x-direction relative to {\tt ekin0pp} \\
{\tt vyn0} & boost velocity y-direction relative to {\tt ekin0pp}  \\
{\tt vzn0} &  boost velocity z-direction relative to {\tt ekin0pp} \\
{\tt nmptheta} &  number of PES measuring points in $\theta$ direction  \\
{\tt nmpphi} &   number of PES measuring points in $\phi$ direction  \\
\hline
\end{tabular}

\begin{tabular}{ll}
\hline
\multicolumn{2}{c}{Namelist {\tt DYNAMIC}} in {\tt for005.<name>} \\
\hline
\multicolumn{2}{c}{\it RTA parameters} \\
\hline
{\tt jrtaint} & modulus for invoking RTAS step \\
{\tt rtamu} &  $\mu$ parameter for quadratic term on $\rho$-constraint\\
{\tt rtamuj} &  $\mu_j$ parameter for quadratic term on $\vec{j}$-constraint \\
{\tt rtasumvar2max} & criterion for maximal variance of s.p. energies \\
{\tt rtaeps} & step size in DCMF iteration\\
{\tt rtae0dmp} & damping energy in DCMF iteration \\
{\tt rtatempinit} & initial temperature in RTA step \\
{\tt rtaforcetemperature} & \textcolor{red}{is that really used?} \\
{\tt rtasigee} & effective electron-electron cross section in RTA \\
{\tt rtars} & effective Wigner-Seitz radius for estimating damping rate \\
\hline
\end{tabular}


\begin{tabular}{ll}
\hline
\multicolumn{2}{c}{Namelist {\tt EXTENSIONS}} in {\tt for005.<name>} \\
\hline
\multicolumn{2}{c}{\it extended options (not to appear in public version)} \\
\hline
{\tt isitmax          }& nr. of imaginary-time steps to improve static
solution
\\
& used with routine {\tt afterburn}, still to be improved\\
{\tt trequest} &  variable checking CPU time to trigger save
operations \\
{\tt timefrac} &  variable checking CPU time to trigger save operations  \\
{\tt iscatterelectron} &  switch to scattering with electron wavepacket\\
{\tt jattach} &  modulus to compute attachement probability\\
{\tt scatterelectronenergy} &  kinetic energy of impinging electron wavepacket\\
{\tt scatterelectronvxn} &  x-velocity  of electron
wavepacket (relative to energy)\\
{\tt scatterelectronvyn} &  y-velocity  of electron
wavepacket (relative to energy) \\
{\tt scatterelectronvzn} &  z-velocity  of electron
wavepacket (relative to energy) \\
{\tt scatterelectronx} & initial x-coordinate of impinging electron \\
{\tt scatterelectrony} & initial y-coordinate of impinging electron  \\
{\tt scatterelectronz} & initial z-coordinate of impinging electron  \\
{\tt scatterelectronw} & initial width  of impinging electron \\
{\tt jescmask} & modulus for detailed print of lost electrons \\
{\tt jescmaskorb} & modulus for state-wise detailed print of lost electrons \\
{\tt eproj}& energy of incoming projectile (= last ion in the list)
\\
{\tt vpx,vpy,vpz}& direction of the incoming projectile
\\
{\tt taccel}& time span over which the projectile is accelerated to
             {\tt eproj}\\
& for {\tt taccel=0} one has to use {\tt init\_lcao=1}\\
{\tt nproj} & element number of atomic projectile \\
{\tt nproj\_states} & nr. of eletronic states in atomic projectile \\
\\
{\tt idenspl} & modulus for printing 2D cuts of density in MTV format\\
{\tt i3dz} &  print z-integrated 2D density, following {\tt idenspl}\\
{\tt i3dx} &  print x-integrated 2D density, following {\tt idenspl}\\
{\tt i3dstate} & print  x- and z-integrated density per state, following {\tt idenspl}\\
{\tt jstboostinv} & modulus for evaluating boost-invariant s.p. energy
and variance\\
\hline
\end{tabular}


\begin{tabular}{ll}
\hline
\multicolumn{2}{c}{Namelist {\tt SURFACE}} in {\tt for005.<name>} \\
\hline
{\tt idielec } & switch to dielectric background  \\
{\tt xdielec } & distance of dielectric background to lowest GSM layer \\
{\tt epsdi } & dielectric constant of  dielectric background \\
{\tt isurf } & switch to activate  dielectric background  \\
\hline
{\tt ne               }& Number of fixed shells in substrate\\
{\tt nc               }& number of O cores in MgO(001)\\
{\tt nk               }& number of Mg cations in MgO(001)\\
{\tt ivdw} & handling of Van-der-Waals with substrate atoms\\
    & 0 $\Longrightarrow$ no VdW\\
    & 1 $\Longrightarrow$ enables full computation of VdW\\
    & 2 $\Longrightarrow$ enables effective VdW through PsP parameters\\
{\tt ifadiadip        }& switch to adiabatic treatment of substrate dipoles\\
{\tt shiftx           }& global shift in $x$ for all substrate atoms\\
{\tt shifty,shiftz    }& as {\tt shiftx} for $y$ and $z$ direction\\
{\tt mion             }& mass of surface anion (16 for O in MgO(001))\\
{\tt mkat             }& mass of surface kation (24.3 for Mg in MgO(001))\\
{\tt me               }& mass of valence shell\\
{\tt cspr             }& spring constant for interaction between core and valence shell\\
{\tt chgc0            }& charge of (anion) core\\
{\tt chge0            }& charge of valence shell\\
{\tt chgk0            }& charge of cation\\
{\tt sigmak           }& gauss width of cation\\
{\tt sigmac           }& gauss width of core\\
{\tt sigmav           }& gauss width of valence shell\\
{\tt iUseCell         }& switch for reading/building lattice of
                          substrate atoms\\
   & 0 $\Longrightarrow$ lattice atoms are read in from input file 'for005surf.*'\\
   & 1 $\Longrightarrow$ lattice is built from replicating unit cell and\\
   &\qquad lattice parameters {\tt rlattvec} ... are
                          read in (see {\tt md.F})\\
{\tt iPotFixed        }& switch for Madelung summation of substrate atoms\\
   & read/write electrostatic potential from particles with imob=0,\\
   &  so that their run-time calculation can be skipped\\
   &   0 $\Longrightarrow$ do not read; calculate full potential at each iteration\\
   &   1 $\Longrightarrow$ read in potFixedIon() from previously prepared file\\
   &  -1 $\Longrightarrow$ calculate potFixedIon() write result to a file which can\\
   &\qquad  be later read in by option 1, stop after that\\
   &   2 $\Longrightarrow$ calculate potFixedIon() at the beginning, do not write\\
{\tt ifmdshort} & includes short range interaction electron--substrate\\
{\tt isrtyp(i,j)      }& type of interaction between the different kinds of particles\\
{\tt                  }& 0   $\rightarrow$ no short range interaction\\
{\tt                  }& 1   $\rightarrow$ GSM core\\
{\tt                  }& 2   $\rightarrow$ GSM valence shell    =1 $\Longrightarrow$ Born-Mayer type\\
{\tt                  }& 3   $\rightarrow$ GSM kation                   =2 $\Longrightarrow$ Argon case\\
{\tt                  }& 4   $\rightarrow$ Na core\\
{\tt                  }& 5   $\rightarrow$ DFT electron\\
{\tt unfixCLateralRadx  }& radius of cylinder with mobile cores \\
{\tt unfixELateralRadx  }& radius of cylinder with mobile valence electrons\\
{\tt fixCBelowx       }&   fixes cores which lay below given x value\\
\hline
\end{tabular}



\begin{tabular}{ll}
\hline
\multicolumn{2}{c}{Namelist {\tt PERIO}} in {\tt for005.<name>} \\
\hline
{\tt ch}& effective charge of ion \\
{\tt amu}& mass of ion in units of hydrogen mass\\
{\tt dr1,dr2}& radii of soft local PsP\\
{\tt prho1,prho2}& strenghts of soft local PsP\\
{\tt crloc}& radius for local part of Goedecker PsP\\
{\tt cc1,cc2}& strengths for local part of  Goedecker PsP\\
{\tt r0g,r1g,r2g}& radii for non-local parts of  Goedecker PsP\\
{\tt h0\_11g,h0\_22g,h0\_33g}& 
  strenghts for non-local parts of  Goedecker PsP\\
{\tt h1\_11g,h1\_22g,h2\_11g}&
  strenghts for non-local parts of  Goedecker PsP\\
{\tt radiong}& carrier radius for projecteor in non-local Goedecker PsP\\
%deprecated option
%{\tt nrow}&
%  ``row'' of element $\longrightarrow$ defines level  of projectors\\
\hline
\end{tabular}


\begin{tabular}{ll}
\hline
\multicolumn{2}{c}{Namelist {\tt FSIC}} in {\tt for005.<name>} \\
\hline
{\tt step}& step size in iteration of localizing or symmetry condition \\
{\tt precis}& precision in iteration of localizing or symmetry condition \\
{\tt SymUtBegin}& nr. iteration where symmetry condition starts\\
 &               for pure localizing step set {\tt  SymUtBegin} $>$ {\tt ismax}\\
{\tt radmaxsym}& limiting value in radius division  for actucal step\\
\hline
\end{tabular}


\vspace*{2cm}

\begin{tabular}{ll}
\hline
\multicolumn{2}{c}{Ionic structure and $e^-$-initialization in {\tt for005ion.<name>}} \\
\hline
& This initialization does not use {\tt NAMELIST} but reads
 input in fixed order.\\
& Each line stands for one ion. Each column has a definite meaning.\\
Col. 1 &  $x$-coordinate \\
Col. 2 &  $y$-coordinate \\
Col. 3 &  $z$-coordinate \\
Col. 4 & 
   number of element in periodic system (e.g.: Na$\leftrightarrow$11)\\
Col. 5 & only {\tt init\_lcao=1}: ordering of nodes in repeated
    initialization at this ion\\
Col. 6 & only {\tt init\_lcao=1}: radius of initial Gaussian at 
    this ion\\
Col. 7 & only {\tt init\_lcao=1}: starting spin for initalization at
    this ion\\
\hline
\end{tabular}
\\[8pt]



\end{document}
