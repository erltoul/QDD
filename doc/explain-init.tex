\documentclass[12pt]{article}
\usepackage[latin2]{inputenc}
%\usepackage{german}
\usepackage{epsfig}
\usepackage{amsmath}
\usepackage{graphicx}

\textwidth 15.8cm
\textheight 22cm
\topmargin -2.5cm
\oddsidemargin 0.0cm
\evensidemargin 0.0cm
\pagestyle{empty}
%\partopsep -33pt
\parsep 12pt
%\topskip -33pt
\parskip 5pt
\parindent 0pt

\renewcommand{\baselinestretch}{1.1}
\newcommand{\onehalf}{\frac{1}{2}}
\renewcommand{\vec}[1]{\mathbf #1}

\unitlength 1mm


\begin{document}

\title{Handling of the cluster 3D Fortran90-code}
\author{Instructions and status reprot}
\date{Started 23 April 2010; status 21. February 2012}
\maketitle

\tableofcontents
\newpage

\section{Installation and usage}

\subsection{Installation}
The following steps assume that you have successfully unpacked
the code and that you are now in the sub-directory 'source\_f90'.

Before compilation, one should update some settings
(for detailed explanations of the parameters see section \ref{sec:inputs}):
\begin{itemize}
\item
 Edit 'define.h' to choose the wanted code options.
\item
 Edit 'params.F90' if you need to change some limiting values
 (rarely required).
\item
 Edit 'makefile' and insert your compiler with its appropriate
 options. Some lines for that are provided and presently commented out.
 Fill the lines and remove the comments if needed.
\item
 Finally execute 'make'. The executable will be copied to the working
 directory which is one level below the sub-directory 'source\_f90'.
 Go back to the working directory. The last file called 'essai*' is
 the new executable.
\end{itemize}



\subsection{Basic input structure}
The cluster 3D code has six entries for options:
\\[-28pt]
\begin{center}
\begin{tabular}{ll}
\hline
\multicolumn{2}{c}{\it compile time}\\
 {\tt define.h} & variants of the code\\
\hline
\multicolumn{2}{c}{\it run time}\\
 {\tt for005.<name>} & general input for settings, static and
 dynamics\\
 {\tt for005ion.<name>} & ionic configuration of cluster\\
 {\tt for005surf.<name>} & atomic configuration of substrate (optional)\\
 {\tt for005} & defines the qualifier {\tt <name>} for 
                the other {\tt for005...} files\\
\hline
\end{tabular}
\end{center}
The first two entries have to be set before compilation. The
other four are read in for an actual run and can be varied from
run to run. The input structure for these files is summarized
in section \ref{sec:inputs}.

\subsection{Some practical advices}

\begin{description}

 \item{Important compile-time settings:}\\
  You have to chose the wnated options in 'define.h'.

 \item{Save and restart:}\\
  The parameters 'isave', 'istat', and 'irest' allow to switch
  saving wavefunctions and restarting from them.\\
  For {\tt ismax>0} and  {\tt isave>1}, 
  the static wavefunctions are saved on {\tt rsave}
  after the static iterations. These can be used in two ways.
  Setting  {\tt istat=1} and  {\tt ismax>0} continues static
  iteration from  {\tt rsave}. Setting  {\tt ismax=0},  {\tt istat=1},
  {\tt irest=0}, and, of course,  {\tt itmax>0} starts a dynamical
  run at time zero with the static wavefunctions from  {\tt rsave}.
  \\
  Dynamical configurations are saved on  {\tt save.<name>} after every
  {\tt isave} time steps. Setting  {\tt irest=1} will
  continue the dymaical calculation 
  from the stage saved in  {\tt save.<name>}.

 \item{Diagonalization amongst occupied states:}\\
  The run time option {\tt ifhamdiag=1}
  activates the diagonalization of the mean-field Hamiltonian
  amongst the active wavefunctions in each static iteration
  step. This option can accelerate the convergence of the static
  solution significantly. {\it However:} At present, this method works 
  safely only if the number of active states {\tt nstate}
  equals the actual number of electrons. This has to be checked 
  by the user. It may work in other cases, but may also induce
  oscillating iteration  which nevers converges.
  

\end{description}

\newpage
\section{Input files}
\label{sec:inputs}


\begin{tabular}{ll}
\hline
\multicolumn{2}{c}{Compile time settings in {\tt define.h}} \\
\hline
\multicolumn{2}{l}{\underline{version control:}}\\
{\tt IVERSION} & define your own version number\\
\hline
\multicolumn{2}{l}{\underline{grid representation of kinetic energy:}}\\
{\tt gridfft} & FFT \\
{\tt findiff} & finite diffences 3. order (yet unsafe)\\
{\tt numerov} & finite diffences 5. order (yet unsafe)\\
\hline
\multicolumn{2}{l}{\underline{Variants of the Coulomb solver (for {\tt gridfft}=1):}}\\
{\tt coufou} & FALR (standard)\\
{\tt coudoub} & exact boundary conditions\\
\hline
\multicolumn{2}{l}{\underline{parallele version:}}\\
{\tt parayes} & use parallelization for wavefunctions \\
{\tt parano}  & produce serial code\\
{\tt simpara} & pseudo-parallel code, runs different inputs
simultaneously\\
\hline
\multicolumn{2}{l}{\underline{versions of SIC for electrons:}}\\
{\tt fullsic} & old full SIC  \\
{\tt symmcond} & old full SIC with double set technique \\
{\tt twostsic} & new full SIC from PhD Messud (obsolete)\\
\hline
\multicolumn{2}{c}{Compile time settings in {\tt define.h} -- part 2} \\
\hline
\multicolumn{2}{l}{\underline{options for substrate:}}\\
{\tt raregas} & enables substrates\\
\hline
\end{tabular}

\newpage

\begin{tabular}{ll}
\hline
\multicolumn{2}{c}{Namelist {\tt GLOBAL}} in {\tt for005.<name>} \\
\hline
\multicolumn{2}{c}{\it choice of system} \\
\hline
{\tt kxbox            }& nr. of grid points in $x$ direction\\
{\tt kybox            }& nr. of grid points in $y$ direction\\
{\tt kzbox            }& nr. of grid points in $z$ direction\\
& box sizes must fulfill {\tt kxbox}$\geq${\tt kybox}$\geq$ {\tt kzbox}\\
{\tt numspin}          & number of spin components (2=full spin treatment)\\
                       & (1=spin averaged, possible problem for ADSIC)\\
{\tt kstate           }& maximum nr. of s.p. states\\
{\tt nclust           }& number of QM electrons\\
{\tt nion             }& number of cluster ions\\
{\tt nspdw            }& number of spin down electrons \\
{\tt nion2            }& selects type of ionic background \\
                       &  0 $\rightarrow$ jellium background \\
                       &  1 $\rightarrow$ background from ionic pseudo-potentials\\
                       &  2 $\rightarrow$ background read in from {\tt potion.dat}\\
{\tt radjel           }& Wigner-Seitz radius of jellium background\\
{\tt surjel         }& surface thickness of jellium background\\
{\tt bbeta         }& quadrupole deformation of jellium background\\
{\tt gamma         }& triaxiality of jellium background\\
{\tt dx,dy,dz,        }& grid spacing (in  Bohr) for the 3D numerical grid\\
&the grid size is defined before compilation in {\tt params.F90}\\
{\tt imob          }& global switch to allow ionic motion (if set to 1) \\
{\tt isurf            }& switch for Ar or MgO surface (isurf=1 activates
          surface\\
{\tt nc               }& number of O cores in MgO(001)\\
{\tt nk               }& number of Mg cations in MgO(001)\\
{\tt rotclustx,y,z } & vector fo angle of initial rotation of ions\\
\hline
\multicolumn{2}{c}{\it initialization of wave functions} \\
\hline
{\tt b2occ            }& deformation for initial harmonic oscillator wf's\\
{\tt gamocc           }& triaxiality for initial harmonic oscillator wf's\\
{\tt deocc            }& shift of inital Fermi energy (determines nr. of
states)\\
{\tt shiftWFx         }& shift of initial wavefunctions in x direction \\
{\tt ishiftCMtoOrigin }& switch to shift center of mass of cluster to origin\\
{\tt ispinsep         }& initialize wavefunctiosn with some spin asymmetry\\
{\tt init\_lcao       }& switches the basis functions to start from\\
& =0 $\Longrightarrow$ harmonic oscillator functions (center can be
moved
  by {\tt shiftWFx})
\\
& =1 $\Longrightarrow$ atomic orbitals = WFs centered at ionic sites
\\
\hline
\multicolumn{2}{c}{\it convergence issues} \\
\hline
{\tt e0dmp            }& damping paramter for static solution of Kohn-Shahm equations\\
& (typically about the energy of the lowest bound state)\\
{\tt epswf            }& step size for static solution of Kohn-Shahm
equations\\
& (of order of 0.5)\\
{\tt epsoro           }& required variance to terminate static iteration\\
&(order of 10$^{-5}$)\\
\hline
\end{tabular}

\newpage

\begin{tabular}{ll}
\hline
\multicolumn{2}{c}{Namelist {\tt DYNAMIC}} in {\tt for005.<name>} \\
\hline
\multicolumn{2}{c}{\it numerical and physical parameters for statics and dynamics} \\
\hline
{\tt dt1              }& time step for propagating electronic wavefunctions\\
{\tt ismax            }& maximum number of static iterations \\
{\tt idyniter         }& switch to s.p. energy as E0DMP for 'iter$>$idyniter'\\
{\tt ifhamdiag} & diagonalization of m.f. Hamiltonian in static step\\
& (presently limited to fully occupied configurations)\\
{\tt itmax            }& number of time steps for electronic propagation\\
{\tt ifexpevol} & exponential evolution 4. order instead of TV splitting\\
{\tt iffastpropag} & accelerated time step in TV splitting\\
  & (for pure electron dynamics, interplay with absorbing b.c. ??)\\
{\tt irest            }& switch to restart dynamics from file 'save'\\
{\tt istat            }& switch to read wavefunctions from file 'rsave'\\
{\tt } &\hspace*{1em}it continues static iteration for 'ismax$>$0' \\
{\tt } &\hspace*{1em}it starts dynamics from these wf's for 'ismax=0' \\
{\tt idenfunc} & choice of density functional for LDA\\
               & 1 $\rightarrow$ Perdew \& Wang 1992 (default setting)\\
               & 2 $\rightarrow$ Gunnarson \& Lundquist\\
               & 3  $\rightarrow$ only exchange in  LDA \\
{\tt isave            }& saves results after every 'isave' steps \\
{\tt                  }& on file 'rsave' in and after static iteration\\
{\tt                  }& on file 'save' in dynamic propagation\\
{\tt ipseudo          }& switch for using pseudo-densities to represent substrate\\
{\tt                  }& atoms \\
{\tt directenergy}   & {\tt .true.} = direct computation of energy \\
                       & (only for {\tt LDA}, {\tt Slater}, {\tt KLI})\\
{\tt ifsicp           }& selects type of self-interaction correction\\
    &  0 = pure LDA, 1 = SIC-GAM, 2 = ADSIC; 3 = SIC-Slater; \\
    &  4 = SIC-KLI; 5 = exact exchange; 6 = old SIC (?);\\
    &  7 = GSlat;  8 = full SIC.\\
    & for activation see switches 
     {\tt kli}, {\tt exchange}, {\tt  fullsic}, {\tt twostsic}.\\
{\tt icooltyp         }& type of cooling (0=none, 1=pseudo-dynamics,\\
{\tt                  }& 2=steepest descent, 3=Monte Carlo)\\
{\tt ifredmas         }& switch to use reduced mass for ions in dynamics\\
{\tt ionmdtyp         }& ionic propagation
                         (0=none, 1=leap-frog, 2=velocity Verlet)\\
{\tt ntref}& nr. time step after which absorbing bounds are deactivated
\\
{\tt nabsorb}          & number of absorbing points on boundary (0 switches off) 
\\
{\tt powabso}          & power of absorbing boundary conditions
\\
{\tt ispherabso}       & switch to spherical mask in absorbing bounds
\\
\hline
\end{tabular}

\begin{tabular}{ll}
\hline
\multicolumn{2}{c}{Namelist {\tt DYNAMIC}} in {\tt for005.<name>} \\
\hline
\multicolumn{2}{c}{\it way of excitation} \\
\hline
{\tt centfx           }& initial boost of electronic wavefuncftions in x-direction\\
{\tt centfy           }& initial boost of electronic wavefuncftions in x-direction\\
{\tt centfz           }& initial boost of electronic wavefuncftions in x-direction\\
{\tt tempion          }& initial temperature of cluster ions \\
{\tt ekmat} & initial kinetic energy of substrate atom (boost in $x$, in eV)\\
{\tt itft   }& choice of shape of laser pulse \\
&   1 = ramp laser pulse, sine switching on/off\\
&   2 = gaussian laser pulse \\
&   3 = cos$^2$ pulse\\
{\tt tnode  }& time (in fs) at which pulse computation starts\\
{\tt deltat }& length of ramp pulse ({\tt itft = 1}), in fs\\
{\tt tpeak  }& time (in fs, relative to {\tt tnode}) at which peak is reached\\
& (for {\tt itft} = 1 and 2, pulse length becomes 2*{\tt tpeak})\\
{\tt omega  }& laser frequency (in Ry)\\
{\tt e0     }& laser field strength in Ry/Bohr\\
{\tt e1x,e1y,e1z   }& orientation of pulse\\
{\tt iexcit} & modus of excitation (0=shifts, 1=rotation)\\
{\tt iangmo} & switch to compute angular momentum\\
{\tt irotat} & axis of rotation for excitation (x=1,y=2,z=2,xyz=4)\\
{\tt phirot} & angle of rotation for excitation (in units of degree)\\
{\tt phangle}        & angle of ``rotation'' into a $1ph$ state\\
{\tt phphase}        & phase of ``rotation'' into a $1ph$ state\\
{\tt nhstate,npstate}& nr. of hole and particle state for $1ph$
                      excitation\\
                     & this $1ph$ option can only be run from {\tt istat=1}
\\
{\tt eproj}& energy of incoming projectile (= last ion in the list)
\\
{\tt vpx,vpy,vpz}& direction of the incoming projectile
\\
{\tt taccel}& time span over which the projectile is accelerated to
             {\tt eproj}
\\
& for {\tt taccel=0} one has to use {\tt init\_lcao=1}
\\
\hline
\end{tabular}

\begin{tabular}{ll}
\hline
\multicolumn{2}{c}{Namelist {\tt DYNAMIC}} in {\tt for005.<name>} \\
\hline
\multicolumn{2}{c}{\it flags for observables} \\
\hline
{\tt iemomsRel        }& calculates multipole momentes of electron density \\
{\tt                  }& relative to origin (0) or c.m. of cluster (1)\\
{\tt istinf           }& modulus for printing information in static iteration \\
{\tt ifspemoms        }& switch to compute and print spatial s.p. moments\\
{\tt iftransme        }& switch to compute and print transition m.elements\\
{\tt ifrhoint\_time   }& switch to slices of integrated densities for all times\\
{\tt jstinf           }& modulus for printing information in dynamic \\
{\tt jinfo            }& modulus for printing dynamical information on {\tt
  infosp.<name>} \\
{\tt jdip             }& modulus for printing dipole moments on {\tt pdip.<name>}\\
{\tt jquad            }& modulus for printing quadrupole moments on {\tt pquad.<name>}\\
{\tt jesc            }& modulus for printing ionization {\tt pescel.<name>}\\
{\tt jenergy          }& modulus for printing energy information on {\tt penergies.<name>} \\
{\tt iflocaliz}        & activates computation of Becke's localization
\\
{\tt jelf}             & modulus for anaylzing and printing electron
localization in dynamics
\\
 & various files are written of the form {\tt pelf*.<name>}
\\
{\tt iflocaliz}             & modulus for anaylzing and printing electron
localization in statics
\\
{\tt jstinf}           & modulus for printing s.p. energies and variances
\\
{\tt jpos}           & modulus for printing ionic positions on {\tt pposion.<name>}
\\
{\tt jvel}           & modulus for printing ionic velocities on {\tt pvelion.<name>}
\\
{\tt jstateoverlap}  & switch to compute overlap of static state
                       with\\
                     & the state directly after dynamical
                       initialization
\\
\hline
\end{tabular}



\begin{tabular}{ll}
\hline
\multicolumn{2}{c}{Namelist {\tt SURFACE}} in {\tt for005.<name>} \\
\hline
{\tt ivdw} & handling of Van-der-Waals with substrate atoms\\
    & 0 $\Longrightarrow$ no VdW\\
    & 1 $\Longrightarrow$ enables full computation of VdW\\
    & 2 $\Longrightarrow$ enables effective VdW through PsP parameters\\
{\tt ifadiadip        }& switch to adiabatic treatment of substrate dipoles\\
{\tt shiftx           }& global shift in $x$ for all substrate atoms\\
{\tt shifty,shiftz    }& as {\tt shiftx} for $y$ and $z$ direction\\
{\tt mion             }& mass of surface anion (16 for O in MgO(001))\\
{\tt mkat             }& mass of surface kation (24.3 for Mg in MgO(001))\\
{\tt me               }& mass of valence shell\\
{\tt cspr             }& spring constant for interaction between core and valence shell\\
{\tt chgc0            }& charge of (anion) core\\
{\tt chge0            }& charge of valence shell\\
{\tt chgk0            }& charge of cation\\
{\tt sigmak           }& gauss width of cation\\
{\tt sigmac           }& gauss width of core\\
{\tt sigmav           }& gauss width of valence shell\\
{\tt iUseCell         }& switch for reading/building lattice of
                          substrate atoms\\
   & 0 $\Longrightarrow$ lattice atoms are read in from input file 'for005surf.*'\\
   & 1 $\Longrightarrow$ lattice is built from replicating unit cell and\\
   &\qquad lattice parameters {\tt rlattvec} ... are
                          read in (see {\tt md.F})\\
{\tt iPotFixed        }& switch for Madelung summation of substrate atoms\\
   & read/write electrostatic potential from particles with imob=0,\\
   &  so that their run-time calculation can be skipped\\
   &   0 $\Longrightarrow$ do not read; calculate full potential at each iteration\\
   &   1 $\Longrightarrow$ read in potFixedIon() from previously prepared file\\
   &  -1 $\Longrightarrow$ calculate potFixedIon() write result to a file which can\\
   &\qquad  be later read in by option 1, stop after that\\
   &   2 $\Longrightarrow$ calculate potFixedIon() at the beginning, do not write\\
{\tt ifmdshort} & includes short range interaction electron--substrate\\
{\tt isrtyp(i,j)      }& type of interaction between the different kinds of particles\\
{\tt                  }& 0   $\rightarrow$ no short range interaction\\
{\tt                  }& 1   $\rightarrow$ GSM core\\
{\tt                  }& 2   $\rightarrow$ GSM valence shell    =1 $\Longrightarrow$ Born-Mayer type\\
{\tt                  }& 3   $\rightarrow$ GSM kation                   =2 $\Longrightarrow$ Argon case\\
{\tt                  }& 4   $\rightarrow$ Na core\\
{\tt                  }& 5   $\rightarrow$ DFT electron\\
{\tt unfixCLateralRadx  }& radius of cylinder with mobile cores \\
{\tt unfixELateralRadx  }& radius of cylinder with mobile valence electrons\\
{\tt fixCBelowx       }&   fixes cores which lay below given x value\\
{\tt iDielec          }& switch to dielectic support\\
{\tt xDielec          }& x below which dielectric zone is activated\\
{\tt epsDi            }& dielectric constant in the dielectric zone\\
\hline
\end{tabular}



\begin{tabular}{ll}
\hline
\multicolumn{2}{c}{Namelist {\tt PERIO}} in {\tt for005.<name>} \\
\hline
{\tt ch}& effective charge of ion \\
{\tt amu}& mass of ion in units of hydrogen mass\\
{\tt dr1,dr2}& radii of soft local PsP\\
{\tt prho1,prho2}& strenghts of soft local PsP\\
{\tt crloc}& radius for local part of Goedecker PsP\\
{\tt cc1,cc2}& strengths for local part of  Goedecker PsP\\
{\tt r0g,r1g,r2g}& radii for non-local parts of  Goedecker PsP\\
{\tt h0\_11g,h0\_22g,h0\_33g}& 
  strenghts for non-local parts of  Goedecker PsP\\
{\tt h1\_11g,h1\_22g,h2\_11g}&
  strenghts for non-local parts of  Goedecker PsP\\
{\tt radiong}& carrier radius for projecteor in non-local Goedecker PsP\\
%deprecated option
%{\tt nrow}&
%  ``row'' of element $\longrightarrow$ defines level  of projectors\\
\hline
\end{tabular}


\begin{tabular}{ll}
\hline
\multicolumn{2}{c}{Namelist {\tt FSIC}} in {\tt for005.<name>} \\
\hline
{\tt step}& step size in iteration of localizing or symmetry condition \\
{\tt precis}& precision in iteration of localizing or symmetry condition \\
{\tt SymUtBegin}& nr. iteration where symmetry condition starts\\
 &               for pure localizing step set {\tt  SymUtBegin}> {\tt ismax}\\
{\tt radmaxsym}& limiting value in radius division  for actucal step\\
\hline
\end{tabular}

\newpage

\appendix
\section{Open ends and to-be-dones}

{\Large\bf Status of Fortran90 code development:}
\begin{itemize}
  \item 
    All {\tt common} blocks have been replaced by modules
    and corresponding {\tt USE} command. The then appearing
    dependences are mapped in the {\tt makefile}.
  \item
    All code is now genuinly double precision and can be
    complied without the {\tt autodouble} option. Only exception
    if the FFT package {\tt fftpack.F90} which still requires
    the  {\tt autodouble}, as handled explicitely in the
    {\tt Makefile}.
    \\
    Note that the precision is set at the header of {\tt params.F90}
    and used as a {\tt KIND} parameter in typical Fortran90 fashion.
  \item
    The somewhat dangerous practice of reusing workspace has been
    abandoned. Workspace is now associated dynamically with
    the {\tt ALLOCATE}/{\tt DEALLOCATE} mechanisms.
  \item
    The compiled code works now for all box sizes and number of s.p.
    states as long as memory allows. The box size and maximum number
    of states is now entered in {\tt for005.<name>} in namelist
    {\tt GLOBAL}.
\end{itemize}

{\Large\bf Next in Fortran90 code development:}
\begin{itemize}
  \item Remove numbered labels and {\tt GOTO} in favour of
        {\tt CYCLE} or {\tt EXIT} switches.
  \item Exploit compact vector operations to simplify
        long (and nested) {\tt DO} loops.
  \item The access {\tt USE kinetic} has been given too generously.
        Confine that to routines which really need it.
  \item The module {\tt params.F90} collects practically all
        global variables. It should be disentangled to more
        specific modules with restricted access. 
  \item There are still problems with running substrates.
        This case has to be tested.
  \item Full SIC has not yet been checked.
  \item The code should be slowly moved to {\tt IMPLICIT NONE}.
\end{itemize}

\newpage

{\Large\bf Open problems of general nature:}
\begin{itemize}
  \item The implementation of GSlat and full SIC needs to be checked
        and updated if necessary.
  \item Check PES and PAD for the option  {\tt parayes}.
  \item Option {\tt iaddCluster} is presently questionable. 
        It may be extended to allow
        for initialization of cluster collisions.
  \item The computation of pseudo-potentials from the substrates
        valence electrons should be separated from the slower
        atomic (ionic) parts. This concerns routine {\tt calcpseudo}.
  \item The setting for the valence-electron mass in 'vstep'
        may be wrong for the case of MgO.
  \item Check proper setting of 'time' in outputs.
  \item Exponential propagation should yet be certified to cooperate
        with ionic motion.
  \item Subgrids for Gaussian pseudo-densities have fixed grid size
        of $\pm 7$ points. This should be made more flexible to
        accommodate mesh size in relation for PsP radius.
  \item Although not necessary for performance, one may replace
        DO loops by the Fortran 95 SUM construct. This will make the
        code more transparent.
  \item Present parallele version  still needs to specify the number
        of nodes at compile time. This should be 
        changed to allow dynamical adjustment of number of nodes.
  \item Spin-averaged code ({\tt numspin}=1) does not reproduce 
        the results from full spin calculation in ase of ADSIC.
        Check.
\end{itemize}





\end{document}
