\documentclass[11pt]{article}

\usepackage[top=2.6cm,bottom=2.6cm,right=2.1cm,left=2.1cm]{geometry}
\usepackage{amsmath}
\usepackage{braket}
\usepackage{color}
\usepackage{multirow}
\usepackage[dvipsnames]{xcolor}
\usepackage[pdftex,colorlinks,citecolor=blue,breaklinks,pdftitle={}]{hyperref}
\usepackage{graphicx}
\usepackage{amsfonts}
\usepackage{float}
\usepackage{tikz}
\usetikzlibrary{shapes,arrows,positioning}
\begin{document}
For the dynamics without RTA.\\
\begin{tabular}{|p{4.5cm}|p{10.2cm}|}
	\hline
	\texttt{energies.<name>} & historical, contains only the binding energy\\
	\hline
	\texttt{forces.<name>} & forces on ions, generated when ion molecular dynamics is active\\
	\hline
	\texttt{<name>.bs} & \textcolor{red}{not found}\\
	\hline
	\texttt{pdip.<name>} &  dipole moment in x, y, z direction, versus time \\
	\hline
	\texttt{penerclu.<name>} & kinetic energy of the cluster in the x,y,z directions and total, versus time, at intervals commanded by the input parameter jener\\
		\hline
	\texttt{pescel.<name>} & proportion of electrons remaining, total number of electrons, number of electrons lost, versus time, at intervals commanded by the input parameter jesc\\
	\hline
	\texttt{plaser.<name>} & laser parameters Ex, Ey, Ez, power, laser energy, etc as a fonction of time\\
	\hline
	\texttt{povlp.<name>} & unused in this version\\
	\hline
	\texttt{penergies.<name>} & Various energies, versus time. The 26 detailed entries (single particle energy, rearrangement energy, etc..) are described in the output file itself. The total energy is at location 18. \\
	\hline
	\texttt{pescOrb.<name>} & Number of electrons lost per orbital, versus time, at intervals commanded by the input parameter jnorms \\
	\hline
	\texttt{pkinenion.<name>} & kinetic energy of the cluster in the x,y,z directions and total, versus time, at intervals commanded by the input parameter jpos\\
	\hline
	\texttt{pPES.<name>} & unused in this version\\
	\hline
	\texttt{pposion.<name>} & positions of the individual ions in x,y,z, and distance to center, versus time, at intervals commanded by the input parameter jpos\\
	\hline
	\texttt{pproba.<name>} & probabilities of charge states versus time, at time intervals commanded by input parameter jnorms\\
	\hline
	\texttt{pprojdip.<name>} & x, y, z pos of projectile versus time, at time intervals commanded by input parameter jdip\\
	\hline
	\texttt{prhov.<name>} & unused in this version\\
	\hline
	\texttt{progstatus} & Only a flag when dynamics are finished\\
	\hline
	\texttt{pspenergies.<name>} & single particle energies versus time, at time intervals commanded by input parameter jinfo\\
	\hline
	\texttt{pspvariances.<name>} & single particle energy variances versus time, at time intervals commanded by input parameter jinfo\\
	\hline
	\texttt{pspvariancesp.<name>} & single particle energy variances versus time (with correction by projection), at time intervals commanded by input parameter jinfo \\
	\hline
	\texttt{ptempion.<name>} & ion temperatures during ionic-core relaxation\\
	\hline
	\texttt{pvelion.<name>} & ion velocities during ionic-core relaxation, or dynamic calculation with molecular dynamics\\
	\hline
	\texttt{rsave.<name>} & This file contains all parameters of a static convergence to allow for a dynamic start without recomputing the statics: to use it set ismax=0 and istat=1  \\
	\hline
	\texttt{save.<name>} & This file contains all parameters to allow for a dynamic start at time : to use it set  irest>=0 \\
	\hline
	\texttt{Time} & Number of points in the calculation box and used wall time to complete the given number of iterations\\
	\hline
\end{tabular}

In this table the files that are generated when RTA is activated. I assume this list will be much shorter.\\
\begin{tabular}{|p{4.5cm}|p{10.2cm}|}
	\hline
	\texttt{convergenceRTA} & \textcolor{red}{ unknown to me, PG?}\\
	\hline
	\texttt{peqstate} &  parameters for convergence of the dtmf process: current iteration number, cycles to convergence, variance, residual err. on density, residual err. on current, parameters mu, muj, energy achieved\\
	\hline
	\texttt{prta} & prints at each rta  step: time, entropy, laser energy and the mu and temperature of a fermi distribution fitted to the occupation numbers\\
	\hline
	\texttt{pspeed.<name>} & prints at each rta step, along x axis, the reference density (spin up and down), achieved density (spin up and down), target x current, achieved x current \\
	\hline
\end{tabular}
\\
\\
Notes: 
\begin{itemize}
\item I do not know \texttt{<name>.bs}
\item I do not know convergenceRTA, maybe this has been added by PG
\item pescel, plaser are for any dynamic problem, not only rta, I moved them
\item some files are missing, but we have a good start here
\end{itemize}
\end{document}



