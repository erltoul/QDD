\documentclass[11pt]{article}
\usepackage{geometry}
\usepackage[utf8]{inputenc}
\usepackage{xcolor}
\usepackage{parskip}
%\usepackage{physics}
\geometry{top=1in,bottom=1in,left=1in,right=1.375in}
\linespread{1.3}

\newcommand{\PGR}[1]{\textcolor{blue}{#1}}


\begin{document}
	\section{Compilation}
		\subsection{NETLIB}
			Dependencies
			\begin{itemize}
				\item Fortran compiler (\texttt{ifort, gfortran, etc.})
			\end{itemize}
			
		\subsection{FFTW}
			Dependencies
			\begin{itemize}
				\item Fortran compiler (\texttt{ifort, gfortran, etc.}), for the main code
				\item C compiler (\texttt{icc, gcc, etc.}), for the FFTW libraries
				\item FFTW libraries
			\end{itemize}
			{\color{red}* \emph{For FFTW to make use of OpenMP, the compiler itself needs to be compiled with OpenMP support! This may sound trivial but e.g. this is not the case for the stock `\texttt{gcc}'} on macOS!}
			
			{\color{red}* \emph{The compiler suite you use to compile QDD has to match the one you used to compile FFTW! E.g., if you used the Intel compiler suite (`\texttt{icc}') to compile FFTW, you MUST use the same suite (`\texttt{ifort}') to compile QDD!}. Actually this only happens in this `direction'. When the FFTW libs are compiled with \texttt{gcc} and the QDD code with \texttt{ifort} everything works fine. Intel-specific problem?}
			
		\subsection{MKL}
			Dependencies
			\begin{itemize}
				\item Fortran compiler (\texttt{ifort, gfortran, etc.}) for the main code
				\item C compiler (\texttt{icc, gcc, etc.}) for the FFTW wrapper library
				\item Intel MKL
				\item Compiled FFTW wrappers to Intel MKL (\texttt{\$MKLROOT/interfaces/fftw3xf} for Fortran)
			\end{itemize}
			
		\subsection{Preprocessor flags}
			Flags from old `\texttt{define.h}'
			\begin{itemize}
				\item \texttt{IVERSION}
                                 \PGR{Allows to communicate a version number}
				\item \texttt{gridfft} 
                                  \PGR{Switches
                                  to Fourier definition of
                                  derivatives. Excludes
                                  \texttt{findiff/numerov}.}
				\item \texttt{findiff}
                                  \PGR{Switches
                                  to 3-point finite differences for
                                  derivatives. Excludes
                                  \texttt{gridfft/numerov}.}
				\item \texttt{numerov}
                                  \PGR{Switches
                                  to 5-point finite differences for
                                  derivatives. Excludes
                                  \texttt{gridfft/findiff}.}
				\item \texttt{coufou}
                                  \PGR{Switches
                                  to Coulomb solver using FAlr.
                                  Excludes \texttt{coudoub}.}
				\item \texttt{coudoub}
                                  \PGR{Switches
                                  to exact Coulomb solver on doubled grid.
                                  Excludes \texttt{coufou}.}
				\item \texttt{coudoub3D}
                                  \PGR{Switches
                                  to 3D FFT in connection with exact
                                  Coulomb solver. 
                                  Requires \texttt{coudoub=1}.}
				\item \texttt{twostsic}
                                  \PGR{Includes all subroutines for
                                    full SIC in compilation.}
				\item \texttt{cmplxsic}
                                  \PGR{Switches to full SIC using
                                    complex wavefunctions. Requires
                                    \texttt{twostsic=1}.} 
				\item \texttt{raregas}
                                  \PGR{Includes all subroutines for
                                    polarizable environment (MgO,
                                    raregases) in compilation.} 
			\end{itemize}
			
			Flags from \texttt{*.F90}-files related to
                        parallelisation \PGR{(all are set in {\tt makefile})}
			\begin{itemize}
				\item \texttt{dynopenmp}
                                   \PGR{Deduced parameter, determined
                                     in {\tt makefile}. Switches to
                                     OpenMP arrangement which
                                     associates CPU
                                     wavefunction-wise. Relevant in
                                     connection with
                                    \texttt{paropenmp=1}.}
				\item \texttt{parano}
                                   \PGR{Switches to compilation for
                                     purely sequential run.
                                  Excludes \texttt{parayes/paropenmp/...}.}
				\item \texttt{parayes}
                                   \PGR{Switches to compilation for
                                     MPI parallel run.
                                  Excludes \texttt{parano/paropenmp/...}.}
				\item \texttt{paraworld}
                                   \PGR{Switches to compilation for
                                     MPI parallel run. Similar to
                                     \texttt{parayes}, but with more
                                     communication statements (??).
                                  Excludes
                                  \texttt{parano/paropenmp/parayes/...}.} 
				\item \texttt{paropenmp}	
                                   \PGR{Switches to compilation for
                                     OpneMp parallel run.
                                  Excludes others as
                                  \texttt{parano/parayes/...}.} 
				\item \texttt{selpara}
                                   \PGR{Obsolete and can be erased.}
				\item \texttt{simpara}		
                                   \PGR{Obsolete and can be erased.}
			\end{itemize}

			Miscellaneous flags from \texttt{*.F90}-files
			\begin{itemize}
				\item \texttt{COMPLEXSWITCH/REALSWITCH}
                                   \PGR{These are internal switches
                                     which are automatically set in
                                     the {\tt makefile}. They serve to
                                   compile independently a REAL and a
                                   COMPLEX version of some
                                   subroutines. The user cannot and
                                   shoult not touch them.}
				\item \texttt{exonly}
                                   \PGR{Obsolete and not actually used.}
				\item \texttt{fftw\_cpu}
                                   \PGR{Deduced parameter, determined
                                     in {\tt makefile}.}
				\item \texttt{fftw\_gpu}
                                   \PGR{Obsolete and must be erased.}
				\item \texttt{fftwnomkl}
                                   \PGR{Deduced parameter, determined
                                     and used in {\tt makefile}.}
				\item \texttt{gunnar}
                                   \PGR{Obsolete and not actually used.}
				\item \texttt{hamdiag}
                                   \PGR{Obsolete and not actually used.}
				\item \texttt{lda\_gpu}
                                   \PGR{Obsolete and must be erased.}
				\item \texttt{locsic}
                                   \PGR{Obsolete and can be erased.}
				\item \texttt{netlib\_fft}
                                   \PGR{Deduced parameter, determined
                                     and used in {\tt makefile}.}
				\item \texttt{oldkinprop}
                                   \PGR{Obsolete and can be erased.}
				\item \texttt{pw92}
                                   \PGR{Obsolete and can be erased.}
				\item \texttt{symmcond}
                                   \PGR{Somewhat mysterious. Seems to
                                     be obsolete together with the
                                     file {\tt symmcond\_step.F90}. Check
                                   performance of full SIC.}
				\item \texttt{tfindiff}
                                    \PGR{Should be replaced by {\tt findiff}.}
				\item \texttt{tnumerov} \PGR{Should be
                                  replaced by {\tt numerov} where
                                  'numerov' means derivates with
                                  5-point precision. But we should
                                  discuss whether we want to maintain
                                  this option.  }
			\end{itemize}
			
			



	
\end{document}
