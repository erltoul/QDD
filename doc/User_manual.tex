%%%%%%%%%%%%%%%%%%%%%%%%%%%%%%%%%%%%%%%%%%%%%%%%%%%%%%%%%%%
%%% PLEASE COMPILE WITH XELATEX TO GET THE FONT SIZES RIGHT
%%%%%%%%%%%%%%%%%%%%%%%%%%%%%%%%%%%%%%%%%%%%%%%%%%%%%%%%%%%
\documentclass[11pt,a4paper]{article}
\usepackage[xetex]{geometry}
\usepackage[utf8]{inputenc}
\usepackage{xcolor}
\usepackage{parskip}
\usepackage{mathtools}
\usepackage{physics}
\usepackage[xetex]{hyperref}
\usepackage{hypertoc}
\usepackage{enumitem}
\usepackage{mathspec} % Use no explicit math shapes (CMU is used)
\usepackage{titlesec,titletoc}
\usepackage[singlelinecheck=false]{caption}
\usepackage{tikz}

\geometry{top=1in,bottom=1in,left=1in,right=1.375in}

\setmathfont(Latin){Minion Pro}
\setmathrm{Minion Pro}
\setmathsf{Myriad Pro Light}
\exchangeforms{phi}
\defaultfontfeatures{
	Ligatures={Required,Common,Contextual,Rare,Historic,TeX},
%	Style=Swash,
	Contextuals={Swash,WordInitial,WordFinal}
}
\setmainfont{Minion Pro}
[
	UprightFeatures={
		SizeFeatures={
			{Size={-8.4},Font=* Caption},
			{Size={8.4-13},Font=*},
			{Size={13-19.9},Font=* Subhead},
			{Size={19.9-},Font=* Display}
		},
	},
	BoldFeatures={
		SizeFeatures={ 
			{Size={-8.4},Font=* Bold Caption},
			{Size={8.4-13},Font=* Bold},
			{Size={13-19.9},Font=* Bold Subhead},
			{Size={19.9-},Font=* Bold Display}
		},
	},
	ItalicFeatures={
		SizeFeatures={ 
			{Size={-8.4},Font=* Italic Caption},
			{Size={8.4-13},Font=* Italic},
			{Size={13-19.9},Font=* Italic Subhead},
			{Size={19.9-},Font=* Italic Display}
		},
	},
	BoldItalicFeatures={
		SizeFeatures={ 
			{Size={-8.4},Font=* Bold Italic Caption},
			{Size={8.4-13},Font=* Bold Italic},
			{Size={13-19.9},Font=* Bold Italic Subhead},
			{Size={19.9-},Font=* Bold Italic Display}
		},
	},
]
\setsansfont{Myriad Pro}
\newfontfamily\headlinefont{Myriad Pro Light}

%\linespread{1.3} % Only touch if the linespacing is insufficient

\definecolor{niceBlue2}{RGB}{16,117,175}
\colorlet{activeColor}{niceBlue2}
\hypersetup{
	colorlinks=true,
	linkcolor=activeColor,
	urlcolor=activeColor,
	citecolor=activeColor,
}\urlstyle{same}

\titleformat{\section}[hang]
	{\headlinefont\LARGE}
	{\thesection}
	{12pt}
	{\color{activeColor}}
\titlespacing{\section}{0em}{4ex}{1ex}
\titleformat{\subsection}[hang]
	{\headlinefont\large}
	{\thesubsection}
	{12pt}
	{\color{activeColor}}
\titlespacing{\subsection}{0em}{4ex}{1ex}
\titleformat{\subsubsection}[hang]
	{\headlinefont\itshape}
	{\thesubsubsection}
	{12pt}
	{\color{activeColor}}
\titlespacing{\subsubsection}{0em}{4ex}{1ex}

\titlecontents{section}[0em]
	{\addvspace{2ex}\headlinefont\large}
	{\hspace*{1.5em}{\contentslabel{1.5em}}}
	{\hspace*{1.5em}{\contentslabel{1.5em}}}
	{\titlerule*[0.7em]{.}\contentspage}
	[]
\titlecontents{subsection}[2.025em]
	{\addvspace{1ex}\headlinefont\normalsize}
	{\hspace*{2em}{\contentslabel{2em}}\color{activeColor}}
	{\hspace*{2em}{\contentslabel{2em}}\color{activeColor}}
	{\titlerule*[0.7em]{.}\contentspage}
	[]
\titlecontents{subsubsection}[4.24em]
	{\headlinefont\itshape\small}
	{\hspace*{3em}{\contentslabel{3em}}}
	{\hspace*{3em}{\contentslabel{3em}}}
	{\titlerule*[0.7em]{.}\contentspage}
	[]

\captionsetup[table]{labelfont={bf,sc,color=activeColor},name=Table}
\captionsetup[figure]{labelfont={bf,color=activeColor},name=Figure}

\renewcommand{\texttt}[1]{{\bfseries\ttfamily #1}}
\usetikzlibrary{shapes}
\newcommand*\circled[1]{
	\tikz[baseline=(char.base)]{
		\node[shape=circle, draw, inner sep=2pt] (char) {#1};
	}
}
\makeatletter
	\newcommand{\showfontsize}{\f@size{} pt}
\makeatother

\title{\fontsize{50}{60}\selectfont{Quantum Dissipative Dynamics}\\\vspace{7ex}\fontsize{50}{60}\selectfont{\textsf{User manual}}\vspace{8ex}}
\author{F.M.G.J. Coppens\\P.-G. Reinhard\\M. Vincendon\\M. Seve Dinh\\E. Suraud}

\begin{document}
	\maketitle
	\thispagestyle{empty}
	\newpage
	\begingroup
		\hypersetup{hidelinks}
		\tableofcontents
	\endgroup
	\addtocontents{toc}{\hfill\small{\sffamily Page}\par}
	\newpage

	\section{Prerequisites}
		To successfully obtain and install the code the following minimum list of requirements has to be obtained
		\begin{itemize}
			\item Internet connection
			\item Git
			\item Fortran compiler
			\item C compiler \textit{(optional)}.
		\end{itemize}	

	\section{Installation}	
		This section will concern itself with how and where to get the code and how to compile it. For the examples treated here the most basic settings are choses so as to minimise the risk of complications. For the full list of compilation parameters and supported libraries, please consult the \textit{QDD Reference Manual}.
		\subsection{Obtaining the code}
			To obtain the code, only one command in a terminal window is required. Change to the directory where you want the top level directory of the code to reside and execute
		\begin{verbatim}
			$ git clone https://github.com/erltls2018/QDD.git
		\end{verbatim}
		This will create a directory `\texttt{QDD}' that contains the software package. The directory
		\begin{verbatim}
			/path/you/chose/QDD/
		\end{verbatim}
		will be referred to in this guide as the `\texttt{\$QDD\_ROOT}'. After the download is complete navigate to the Fortran source directory:
		\begin{verbatim}
			$ cd $QDD_ROOT/src/qdd
		\end{verbatim}
		
		\subsection{Compilation}
			To get everything up and running as easy and fast as possible we will compile the QDD package with the default `\texttt{Makefile}'. This means QDD will be using:
		\begin{itemize}
			\item Fast Fourier transforms from the \textit{Netlib FFTPACK}
			\item The \textit{GNU Fortran} compiler \texttt{gfortran}, that can be downloaded here:\\ \url{https://gcc.gnu.org/wiki/GFortran}
			\item No OpenMP parallelisation
			\item No MPI parallelisation
		\end{itemize}
		Different compilers, Fast Fourier libraries and parallelisation options can be chosen as well. This information can be found in the~\textit{QDD Reference Manual}.
		
		If all prerequisites are met, simply execute
		\begin{verbatim}
			$ make
		\end{verbatim}
		After the build process is finished the executable `\texttt{qdd}' will be in the ``bin''-directory
		\begin{verbatim}
			$ $QDD_ROOT/bin/qdd
		\end{verbatim}

	\section{Basic I/O structure of a ground state calculation}
		
		\subsection{Input files \& input parameters}
		Each calculation will have its own directory and in it its own set of input files where all the details and the initial conditions of the system to be calculated are set. While the calculation is running, screen output and output files are generated that give information about the current state of the system while it is running and also about the final state when the calculation is finished. There are a minimum of 3 files required to start a calculation. They are listed in Table~\ref{tab:input-files}.
		
		\begin{table}[t]
			\caption{Minimum set of input files.}\label{tab:input-files}
			\begin{tabular}{|p{3.5cm}|p{11.2cm}|}
				\hline
				\texttt{for005} & top level file containing the calculation identifier `\texttt{<name>}'. This can be any string of characters, e.g. `\texttt{h2o}' or `\texttt{na8}'. \\
			  	\hline
			  	\texttt{for005.<name>} & top level file containing the main parameters of the calculation using Fortran's \textit{namelist}-mechanism. For description of all these input parameters see Tables~\ref{tab:input-params-sys-choice},~\ref{tab:input-params-wfs},~\ref{tab:input-params-conv},~\ref{tab:dyn-input-params-general},~\ref{tab:dyn-input-params-excitation},~\ref{tab:dyn-input-params-observables},~and~\ref{tab:dyn-input-params-rta}.\\
			  	\hline
			  	\texttt{for005ion.<name>} & top level file containing the locations and types of the ions.\\
			 	\hline
			\end{tabular}
		\end{table}
		
		For the input file `\texttt{for005.<name>}' there is a minimum set of input parameters. The ones related to a static calculation are listed and explained in Tables~\ref{tab:input-params-sys-choice},~\ref{tab:input-params-wfs}~and~\ref{tab:input-params-conv}. A complete list of all the input parameters can be found in the \textit{QDD Reference Manual}.
		
		The input file `\texttt{for005ion.<name>}' contains information about the ion cores. Each line in the file represents one ion core and each line is composed of several fields like so
		\begin{align*}
			\underbrace{\tikz[baseline]{\node[fill=blue!20,anchor=base]{$x_n\:\:\: y_n\:\:\: z_n$};}}_{\circled{1}}\quad
			\underbrace{\tikz[baseline]{\node[fill=red!20,anchor=base]{$cen_n$};}}_{\circled{2}}\quad
			\underbrace{\tikz[baseline]{\node[fill=green!20,anchor=base]{$ijk$};}}_{\circled{3}}\quad
			\underbrace{\tikz[baseline]{\node[fill=yellow!20,anchor=base]{$r_n$};}}_{\circled{4}}\quad
			\underbrace{\tikz[baseline]{\node[fill=magenta!20,anchor=base]{$\mathcal{S}_n$};}}_{\circled{5}}
		\end{align*}
		\begin{itemize}
			\item[\circled{1}] are the $(x,y,z)$-coordinates of ion core $n$
			\item[\circled{2}] is the chemical element number in the periodic table of ion-core $n$
			\item[\circled{3}] is the node ordering in repeat initialisation, where $(i,j,k)\in\{x,y,x\}$
			\item[\circled{4}] is radius of the initial Gaussian around ion-core $n$
			\item[\circled{5}] is start spin for initialisation at ion-core $n$
		\end{itemize}\newpage

		For example, an ion input file for H$_2$O (\emph{say} \texttt{for005ion.H2O}) could look like

		\tikz[baseline]{\node[fill=blue!20,anchor=base]
			{\texttt{0.22835~~0.00000~~0.00000}};
		}
		\tikz[baseline]{\node[fill=red!20,anchor=base]
			{\texttt{8}};
		}
		\tikz[baseline]{\node[fill=green!20,anchor=base]
			{\texttt{xyz}};
		}
		\tikz[baseline]{\node[fill=yellow!20,anchor=base]
			{\texttt{1.0}};
		}
		\tikz[baseline]{\node[fill=magenta!20,anchor=base]
			{\texttt{~1}};
		}
		
		\tikz[baseline]{\node[fill=blue!20,anchor=base]
			{\texttt{-0.91350~~1.47420~0.00000}};
		}
		\tikz[baseline]{\node[fill=red!20,anchor=base]
			{\texttt{1}};
		}
		\tikz[baseline]{\node[fill=green!20,anchor=base]
			{\texttt{xyz}};
		}
		\tikz[baseline]{\node[fill=yellow!20,anchor=base]
			{\texttt{1.0}};
		}
		\tikz[baseline]{\node[fill=magenta!20,anchor=base]
			{\texttt{-1}};
		}
		
		\tikz[baseline]{\node[fill=blue!20,anchor=base]
			{\texttt{-0.91350~-1.47420~0.00000}};
		}
		\tikz[baseline]{\node[fill=red!20,anchor=base]
			{\texttt{1}};
		}
		\tikz[baseline]{\node[fill=green!20,anchor=base]
			{\texttt{xyz}};
		}
		\tikz[baseline]{\node[fill=yellow!20,anchor=base]
			{\texttt{1.0}};
		}
		\tikz[baseline]{\node[fill=magenta!20,anchor=base]
			{\texttt{-1}};
		}		

		\begin{table}[t]
			\caption{System choice definitions in the `GLOBAL' namelist in \texttt{for005.<name>}}\label{tab:input-params-sys-choice}
			\begin{tabular}{|p{3.5cm}|p{11.2cm}|}
				\hline
				\multicolumn{2}{|c|}{The \texttt{GLOBAL} namelist}\\
				\hline
				\multicolumn{2}{|c|}{\textit{\color{activeColor}concerning system choices}}\\
				\hline
				\texttt{kxbox}, & number of grid points in the $\{x,y,z\}$-direction.\\
				\texttt{kybox}, & A typical value is 64 points in each direction.\\
				\texttt{kzbox} & The box sizes must fulfil the condition: \texttt{kxbox} $\geq$ \texttt{kybox} $\geq$ \texttt{kzbox}.\\
				\hline
				\texttt{kstate}& maximum number of possible single-particle (s.p.) states (can be larger than \texttt{nclust})\\
			  	\hline
				\texttt{numspin}& number of spin components\\
				& 1 $\rightarrow$ spin averaged (possible problem for ADSIC)\\
				& 2 $\rightarrow$ full spin treatment)\\
				\hline
				\texttt{nclust}& number of QM electrons; if set to 0 or a negative value (charge) this will be automatically calculated: \\
				& \texttt{nclust} = $\sum_{i=1}^{n_{ion}} Z_{ion} = \mathrm{charge}$, where $Z_{ion}$ is the charge of each ion\\
				\hline
				\texttt{nion}& number of cluster ions\\
				\hline
				\texttt{nspdw}& number of spin down electrons \\
				\hline
				\texttt{nion2}& selects type of ionic background \\
				                       &  0 $\rightarrow$ jellium background \\
				                       &  1 $\rightarrow$ background from ionic pseudo-potentials\\
				                       &  2 $\rightarrow$ background read in from \texttt{potion.dat}\\
				\hline
				\texttt{radjel           }& Wigner-Seitz radius of jellium background\\
				\hline
				\texttt{surjel         }& surface thickness of jellium background\\
				\hline
				\texttt{bbeta         }& quadrupole deformation of jellium background\\
				\hline
				\texttt{gamma         }& triaxiality of jellium background\\
				\hline
				\texttt{dx,dy,dz        }& grid spacing (in  Bohr) for the 3D numerical grid. If negative, this will be set to an optimal value, a value for \texttt{kxbox}  will be suggested in the file `\texttt{nx}', the code stops and has to be restarted. The grid size is defined before compilation in \texttt{params.F90} and it has to correlate with the pseudo-potentials corresponds to ecut in solid state\\
				\hline
				\texttt{imob}& global switch to allow ionic motion (if set to 1) \\
				\hline
				\texttt{isurf}& switch for Ar or MgO surface (isurf=1 activates surface)\\
				\hline
				\texttt{iDielec}& switch to dielectic support\\
				\hline
				\texttt{xDielec}& x below which dielectric zone is activated\\
				\hline
				\texttt{epsDi}& dielectric constant in the dielectric zone\\
				\hline
				\texttt{rotclustx,y,z} & vector fo angle of initial rotation of ions\\
				\hline
			\end{tabular}
		\end{table}

		\begin{table}[t]
			\caption{Wave function initialisation in the GLOBAL namelist in \texttt{for005.<name>}}\label{tab:input-params-wfs}
			\begin{tabular}{|p{3.5cm}|p{11.2cm}|}
				\hline
				\multicolumn{2}{|c|}{The \texttt{GLOBAL} namelist}\\
				\hline
				\multicolumn{2}{|c|}{\textit{\color{activeColor}concerning the initialisation of wave functions}} \\
				\hline
				\texttt{b2occ}& deformation for initial harmonic oscillator wf's\\
				\hline
				\texttt{gamocc}& triaxiality for initial harmonic oscillator wf's\\
				\hline
				\texttt{deocc}& shift of initial Fermi energy (determines nr. of
				states)\\
				\hline
				\texttt{shiftWFx}& shift of initial wave functions in $x$-direction \\
				\hline
				\texttt{ishiftCMtoOrigin}& switch to shift centre of mass of cluster to origin\\
				\hline
				\texttt{ispinsep}& initialise wave functions with some spin asymmetry\\
				\hline
				\texttt{init\_lcao}& choice of basis for wave function initialisation \\
				& 0 $\rightarrow$ harmonic oscillator functions (centre can be
				moved by \texttt{shiftWFx}) \\
				& 1 $\rightarrow$ atomic orbitals: wave functions are centred at ionic sites\\
				\hline
			\end{tabular}
		\end{table}

		\begin{table}[t]
			\caption{Convergence parameters in the GLOBAL namelist in \texttt{for005.<name>}}\label{tab:input-params-conv}
			\begin{tabular}{|p{3.5cm}|p{11.2cm}|}
				\hline
				\multicolumn{2}{|c|}{The \texttt{GLOBAL} namelist}\\
				\hline
				\multicolumn{2}{|c|}{\textit{\color{activeColor}concerning convergence issues}}\\
				\hline
				\texttt{e0dmp}& damping parameter for static solution of Kohn-Sham equations (typically about the energy of the lowest bound state)\\
				\hline
				\texttt{epswf}& step size for static solution of Kohn-Shahm equations (of order of 0.5)\\
				\hline
				\texttt{epsoro}& required variance to terminate static iteration (of order 10$^{-5}$)\\
				\hline
			\end{tabular}
		\end{table}
		
		Examples of these input files can be found in `\texttt{\$QDD\_ROOT/examples/}'. In the following sections, example calculations of the covalent molecule water will be treated in ever increasing complexity.
			
		\subsection{Output files}
			During a calculation output files are generated and stored in the same directory as where the \texttt{qdd} binary is executed. For a ground state calculation the output files are listed in Table~\ref{tab:static-output-files}.
			\begin{table}[t]
				\caption{Output files generated during a \emph{static} calculation}\label{tab:static-output-files}
				\begin{tabular}{|p{3.5cm}|p{11.2cm}|}
					\hline
					\texttt{dx}& grid spacing in units of Bohr. \uppercase{not sure about this. the value in this file seems to conflict with the value set in the input file} \\
					\hline
					\texttt{nx}& this file contains a suggested value for \texttt{kxbox} when \texttt{dx} is set to a negative value\\
					\hline
					\texttt{for006.0<name>}& protocol file.\\
					\hline
					\texttt{infosp.<name>}& energy and variances at given iteration numbers determined by the variable \texttt{jinfo} in the \texttt{DYNAMIC} namelist.\\
					\hline
					\texttt{poptions.<name>}& this file contains an overview of the chosen options on solvers, compiler options, etc.\\
					\hline
					\texttt{pstat.<name>}& contains the final information about the single particle energies, spins, variances, occupation numbers, monopole-, dipole- and quadrupole moments, etc.\\
					\hline
				\end{tabular}
			\end{table}
		
	\section{An illustrative ground state calculation: H$_\mathsf{2}$O}
			
		The input files for the water molecule can be found in 
		\begin{verbatim}
			$QDD_ROOT/samples/H2O/
		\end{verbatim}
		As discussed in the previous section the 3 files that you will find here are `\texttt{for005}', `\texttt{for005.h2o}' and `\texttt{for005ion.h2o}' respectively. You can either copy these files to a location of your choosing, or simply run the `\texttt{qdd}' executable in each directory by executing
		\begin{verbatim}
			$ cd /path/to/the/for005*/files
			$ $QDD_ROOT/bin/qdd > terminal.out 2> error.out
		\end{verbatim}
		This will save the terminal output on the screen to the file `\texttt{terminal.out}' and any errors that might be generated during the calculation to `\texttt{error.out}'. Depending on the speed of your machine, after a few minutes you should have a file listing similar to this
		\begin{verbatim}
			-rw------- 1 coppens   25 Feb  1 11:55 dx
			-rw------- 1 coppens   34 Feb  1 11:57 error.out
			-rw------- 1 coppens    4 Feb  1 11:52 for005
			-rw------- 1 coppens 1.2K Feb  1 11:53 for005.H2O
			-rw------- 1 coppens  160 Nov 28 11:12 for005ion.H2O
			-rw------- 1 coppens  46K Feb  1 11:57 for006.0H2O
			-rw------- 1 coppens 2.0K Feb  1 11:57 infosp.H2O
			-rw------- 1 coppens   13 Feb  1 11:55 nx
			-rw------- 1 coppens 1.3K Feb  1 11:55 poptions.H2O
			-rw------- 1 coppens 2.3K Feb  1 11:57 pstat.H2O
			-rw------- 1 coppens  41M Feb  1 11:57 rsave.H2O
			-rw------- 1 coppens 138K Feb  1 11:57 terminal.out
		\end{verbatim}
		In the file `\texttt{pstat.H2O}' you will find  
		\begin{verbatim}
			final protocol of static for IFSICP=  2
			level:  1  spin,occup,ekin,esp,variance =  1  1.00000  1.31912 -2.33723  5.8472E-10
			level:  2  spin,occup,ekin,esp,variance =  1  1.00000  2.51534 -1.59199  5.3699E-10
			level:  3  spin,occup,ekin,esp,variance =  1  1.00000  2.46317 -1.25959  5.3194E-10
			level:  4  spin,occup,ekin,esp,variance =  1  1.00000  2.74713 -1.11479  5.7191E-10
			level:  5  spin,occup,ekin,esp,variance =  1  0.00000  0.71328 -0.42203  7.7313E-07
			level:  6  spin,occup,ekin,esp,variance =  1  0.00000  0.73195 -0.26481  2.9550E-03
			level:  7  spin,occup,ekin,esp,variance =  1  0.00000  0.30830 -0.21468  5.6298E-03
			level:  8  spin,occup,ekin,esp,variance =  1  0.00000  0.31976 -0.21151  6.3903E-03
			level:  9  spin,occup,ekin,esp,variance =  1  0.00000  0.35431 -0.15832  7.9917E-03
			level: 10  spin,occup,ekin,esp,variance =  1  0.00000  0.20732 -0.12894  1.4731E-03
			level: 11  spin,occup,ekin,esp,variance = -1  1.00000  1.31912 -2.33723  5.8472E-10
			level: 12  spin,occup,ekin,esp,variance = -1  1.00000  2.51534 -1.59199  5.3699E-10
			level: 13  spin,occup,ekin,esp,variance = -1  1.00000  2.46317 -1.25959  5.3194E-10
			level: 14  spin,occup,ekin,esp,variance = -1  1.00000  2.74713 -1.11479  5.7191E-10
			level: 15  spin,occup,ekin,esp,variance = -1  0.00000  0.71328 -0.42203  7.7313E-07
			level: 16  spin,occup,ekin,esp,variance = -1  0.00000  0.73195 -0.26481  2.9550E-03
			level: 17  spin,occup,ekin,esp,variance = -1  0.00000  0.30830 -0.21468  5.6298E-03
			level: 18  spin,occup,ekin,esp,variance = -1  0.00000  0.31976 -0.21151  6.3903E-03
			level: 19  spin,occup,ekin,esp,variance = -1  0.00000  0.35431 -0.15832  7.9917E-03
			level: 20  spin,occup,ekin,esp,variance = -1  0.00000  0.20732 -0.12894  1.4731E-03
			binding energy  =  -32.6355092
			total variance  =  5.5684E-10
			sp pot, sp kin, rearr, nonlocal=  -30.69673   18.08954   -1.06211    0.00000
			e_coul: i-i , e-i , e-e , total=   13.54907  -97.56193   39.39447  -44.61839
			mon.:   8.00
			dip.in  :    0.00000    0.00000    0.00000
			dip.out :   -0.06362    0.00000    0.00000
			quadrupole moments:
			xx,yy,zz:     0.9037     0.9913     0.7949
			xy,zx,zy:     0.0000     0.0000     0.0000
			spindip.:    -0.0000     0.0000    -0.0000
			omegam,rhops,N_el,rhomix:     0.0000     0.0000     0.0000     0.0000
		\end{verbatim}
		
		\subsection{Results and observables}
			Other than the terminal output there are also a number of other files created during and when the calculation is finished.
			\subsubsection{Total binding energy}
			\subsubsection{Ionisation potential}
			\subsubsection{Eigen energies of the single electron orbitals}
			\subsubsection{Monopole-, dipole- and quadrupole moments}
			\subsubsection{H.O.M.O-L.U.M.O. gap}
			\subsubsection{Potentials and electron density}
			\subsubsection{Structure information}

		
%		\subsection{Metal clusters: Na$_\mathsf{8}$}
%			One would expect on purely classical grounds that the lowest energy configuration will be an equilateral triangle. As the calculation will show, this is not the case; the triangle is slightly more opened. This shows that one needs to included  quantum effects to explain the final structure.
			
		\subsection{Relaxation of the ion-cores positions using pseudo-dynamics}

	\section{Basic I/O structure of dynamic calculations}
		\subsection{Input parameters in the \texttt{DYNAMIC} namelist}
		
			\begin{table}[t]
				\caption{Dynamical paramaters in the DYNAMIC namelist in \texttt{for005.<name>}}\label{tab:dyn-input-params-general}
				\begin{tabular}{|p{3.5cm}|p{11.2cm}|}
					\hline
					\multicolumn{2}{|c|}{The \texttt{DYNAMIC} namelist}\\
					\hline
					\multicolumn{2}{|c|}{\textit{\color{activeColor}numerical and physical parameters for statics and dynamics}} \\
					\hline
					\texttt{dt1}& time step for propagating electronic wave functions,  $\frac{\Delta t}{\Delta x^{2}}\leq 1$\\
					\hline
					\texttt{ismax}& maximum number of static iterations \\
					\hline
					\texttt{idyniter}& switch to s.p. energy as E0DMP for 'iter$>$idyniter'\\
					\hline
					\texttt{ifhamdiag} & diagonalization of m.f. Hamiltonian in static step (presently limited to fully occupied configurations)\\
					\texttt{isitmax}& nr. of imaginary-time steps to improve static solution\\
					\hline
					\texttt{itmax}& number of time steps for electronic propagation\\
					\hline
					\texttt{ifexpevol} & exponential evolution 4. order instead of TV splitting\\
					\hline
					\texttt{iffastpropag} & accelerated time step in TV splitting (for pure electron dynamics, interplay with absorbing b.c. ??)\\
					\hline
					\texttt{irest}& switch to restart dynamics from file 'save'\\
					\hline
					\texttt{istat}& switch to read wavefunctions from file 'rsave'
					\begin{itemize}
						\item it continues static iteration for 'ismax$>$0'
						\item it starts dynamics from these wf's for 'ismax=0'
						\vspace{-3ex}
					\end{itemize}\\
					\hline
					\texttt{idenfunc} & choice of density functional for LDA\\
						& 1 $\rightarrow$ Perdew \& Wang 1992 (default setting)\\
						& 2 $\rightarrow$ Gunnarson \& Lundquist\\
						& 3 $\rightarrow$ only exchange in  LDA \\
					\hline
					\texttt{isave}& saves results after every 'isave' steps \\
					\texttt{}& on file 'rsave' in and after static iteration\\
					\texttt{}& on file 'save' in dynamic propagation\\
					\hline
					\texttt{ipseudo}& switch for using pseudo-densities to represent substrate\\
					\texttt{}& atoms \\
					\hline
					\texttt{ipsptype}& type of pseudopotentials: 0 = soft local (errf);\\
						& 1 = full Goedecker; 2 = local Goedecker;\\
						& 3 = read from file goed.asci (no need to specify)  ;\\
						& 4 = semicore read from file goed.asci\\
					\hline
					\texttt{directenergy}   & \texttt{.true.} = direct computation of energy \\
						& (only for \texttt{LDA}, \texttt{Slater}, \texttt{KLI})\\
					\hline
					\texttt{ifsicp}& selects type of self-interaction correction\\
						&  0 = pure LDA, 1 = SIC-GAM, 2 = ADSIC; 3 = SIC-Slater; \\
						&  4 = SIC-KLI; 5 = exact exchange; 6 = inactive;\\
						&  7 = localized SIC;  8 = full SIC (double set).\\
						& IFSICP=7 or 8 requires switch \texttt{twostsic=1} in \texttt{define.h}.\\
						& Option IFSICP=7 needs yet testing.\\
					\hline
					\texttt{icooltyp}& type of cooling (0=none, 1=pseudo-dynamics,\\
					\texttt{}& 2=steepest descent, 3=Monte Carlo)\\
					\hline
					\texttt{ifredmas}& switch to use reduced mass for ions in dynamics\\
					\hline
					\texttt{ionmdtyp}& ionic propagation (0=none, 1=leap-frog, 2=velocity Verlet)\\
					\hline
					\texttt{ntref}& nr. time step after which absorbing bounds are deactivated\\
					\hline
					\texttt{nabsorb}& number of absorbing points on boundary (0 switches off)\\
					\hline
					\texttt{powabso}& power of absorbing boundary conditions\\
					\hline
					\texttt{ispherabso}& switch to spherical mask in absorbing bounds\\
					\hline
				\end{tabular}
			\end{table}

			\begin{table}[t]
				\caption{Dynamical paramaters in the DYNAMIC namelist in \texttt{for005.<name>}}\label{tab:dyn-input-params-excitation}
				\begin{tabular}{|p{3.5cm}|p{11.2cm}|}
					\hline
					\multicolumn{2}{|c|}{The \texttt{DYNAMIC} namelist}\\
					\hline
					\multicolumn{2}{|c|}{\textit{\color{activeColor}way of excitation}} \\
					\hline
					\texttt{centfx}& initial boost of electronic wavefuncftions in x-direction\\
					\hline
					\texttt{centfy}& initial boost of electronic wavefuncftions in y-direction\\
					\hline
					\texttt{centfz}& initial boost of electronic wavefuncftions in z-direction\\
					\hline
					\texttt{tempion}& initial temperature of cluster ions \\
					\hline
					\texttt{ekmat} & initial kinetic energy of substrate atom (boost in $x$, in eV)\\
					\hline
					\texttt{itft}& choice of shape of laser pulse \\
						&   1 = ramp laser pulse, sine switching on/off\\
						&   2 = gaussian laser pulse \\
						&   3 = cos$^2$ pulse\\
					\hline
					\texttt{tnode}& time (in fs) at which pulse computation starts\\
					\hline
					\texttt{deltat}& length of ramp pulse (\texttt{itft = 1}), in fs\\
					\hline
					\texttt{tpeak}& time (in fs, relative to \texttt{tnode}) at which peak is reached\\
						& (for \texttt{itft} = 1 and 2, pulse length becomes 2*\texttt{tpeak})\\
					\hline
					\texttt{omega}& laser frequency (in Ry)\\
					\hline
					\texttt{e0}& laser field strength in Ry/Bohr\\
					\hline
					\texttt{e1x,e1y,e1z}& orientation of pulse\\
					\hline
					\texttt{e0\_2}& field strength of second laser pulse (only \texttt{itft=3}) \\
					\hline
					\texttt{phase2}& phase of second pulse \\
					\hline
					\texttt{omega2}& frequency of second pulse\\
					\hline
					\texttt{tstart2}& initial ime of second pulse\\
					\hline
					\texttt{tpeak2} & peak time of 2. pulse (pulse length is \texttt{2*tpeak2})\\
					\hline
					\texttt{iexcit} & modus of excitation (0=shifts, 1=rotation)\\
					\hline
					\texttt{iangmo} & switch to compute angular momentum\\
					\hline
					\texttt{irotat} & axis of rotation for excitation (x=1,y=2,z=2,xyz=4)\\
					\hline
					\texttt{phirot} & angle of rotation for excitation (in units of degree)\\
					\hline
					\texttt{phangle} & angle of ``rotation'' into a $1ph$ state\\
					\hline
					\texttt{phphase} & phase of ``rotation'' into a $1ph$ state\\
					\hline
					\texttt{nhstate,npstate}& nr. of hole and particle state for $1ph$ excitation\\
						& this $1ph$ option can only be run from \texttt{istat=1}\\
					\hline
					\texttt{eproj}& energy of incoming projectile (= last ion in the list)\\
					\hline
					\texttt{vpx,vpy,vpz}& direction of the incoming projectile\\
					\hline
					\texttt{taccel}& time span over which the projectile is accelerated to \texttt{eproj}\\
						& for \texttt{taccel=0} one has to use \texttt{init\_lcao=1}\\
					\hline
				\end{tabular}
			\end{table}


			\begin{table}[t]
				\caption{Parameters that control observables and output in the DYNAMIC namelist in \texttt{for005.<name>}}\label{tab:dyn-input-params-observables}
				\begin{tabular}{|p{3.5cm}|p{11.2cm}|}
					\hline
					\multicolumn{2}{|c|}{The \texttt{DYNAMIC} namelist}\\
					\hline
					\multicolumn{2}{|c|}{\textit{\color{activeColor}flags for observables}} \\
					\hline			
					\texttt{iemomsRel}& calculates multipole momentas of electron density relative to origin (0) or c.m. of cluster (1)\\
					\hline
					\texttt{istinf}& modulus for printing information in static iteration \\
					\hline
					\texttt{ifspemoms}& switch to compute and print spatial s.p. moments\\
					\hline
					\texttt{iftransme}& switch to compute and print transition m.elements\\
					\hline
					\texttt{ifrhoint\_time}& switch to slices of integrated densities for all times\\
					\hline
					\texttt{jstinf}& modulus for printing information in dynamic \\
					\hline
					\texttt{jinfo}& modulus for printing dynamical information on \texttt{infosp.<name>} \\
					\hline
					\texttt{jdip}& modulus for printing dipole moments on \texttt{pdip.<name>}\\
					\hline
					\texttt{jquad}& modulus for printing quadrupole moments on \texttt{pquad.<name>}\\
					\hline
					\texttt{jesc}& modulus for printing ionization \texttt{pescel.<name>}\\
					\hline
					\texttt{jenergy}& modulus for printing energy information on \texttt{penergies.<name>} \\
					\hline
					\texttt{iflocaliz}& activates computation of Becke's localisation\\
					\hline
					\texttt{jelf}& modulus for analysing and printing electron localisation in dynamics various files are written of the form \texttt{pelf*.<name>}\\
					\hline
					\texttt{iflocaliz}& modulus for analysing and printing electron localisation in statics\\
					\hline
					\texttt{jstinf}& modulus for printing s.p. energies and variances\\
					\hline
					\texttt{jpos}& modulus for printing ionic positions on \texttt{pposion.<name>}\\
					\hline
					\texttt{jvel}& modulus for printing ionic velocities on \texttt{pvelion.<name>}\\
					\hline
					\texttt{jstateoverlap}& switch to compute overlap of static state with the state directly after dynamical initialisation\\
					\hline
				\end{tabular}
			\end{table}

		\subsection{Output files}
			\begin{table}[t]
				\caption{Output files generated during a \emph{dynamic} calculation}\label{tab:dynamic-output-files}
				\begin{tabular}{|p{4.5cm}|p{10.2cm}|}
					\hline
					\texttt{energies.<name>} & historical, contains only the binding energy\\
					\hline
					\texttt{forces.<name>} & forces on ions, generated when ion molecular dynamics is active\\
					\hline
					\texttt{<name>.bs} & output suited for further processing by freeware which  can make 3D structure plots of molecular configurations in the typical chemical style\\
					\hline
					\texttt{pdip.<name>} &  dipole moment in x, y, z direction, versus time \\
					\hline
					\texttt{penerclu.<name>} & kinetic energy of the cluster in the x,y,z directions and total, versus time, at intervals commanded by the input parameter jener\\
					\hline
					\texttt{pescel.<name>} & proportion of electrons remaining, total number of electrons, number of electrons lost, versus time, at intervals commanded by the input parameter jesc\\
					\hline
					\texttt{plaser.<name>} & laser parameters Ex, Ey, Ez, power, laser energy, etc as a fonction of time\\
					\hline
					\texttt{povlp.<name>} & unused in this version\\
					\hline
					\texttt{penergies.<name>} & Various energies, versus time. The 26 detailed entries (single particle energy, rearrangement energy, etc..) are described in the output file itself. The total energy is at location 18. \\
					\hline
					\texttt{pescOrb.<name>} & Number of electrons lost per orbital, versus time, at intervals commanded by the input parameter jnorms \\
					\hline
					\texttt{pkinenion.<name>} & kinetic energy of the cluster in the x,y,z directions and total, versus time, at intervals commanded by the input parameter jpos\\
					\hline
					\texttt{pPES.<name>} & unused in this version\\
					\hline
					\texttt{pposion.<name>} & positions of the individual ions in x,y,z, and distance to center, versus time, at intervals commanded by the input parameter jpos\\
					\hline
					\texttt{pproba.<name>} & probabilities of charge states versus time, at time intervals commanded by input parameter jnorms\\
					\hline
					\texttt{pprojdip.<name>} & x, y, z pos of projectile versus time, at time intervals commanded by input parameter jdip\\
					\hline
					\texttt{prhov.<name>} & unused in this version\\
					\hline
					\texttt{progstatus} & Only a flag when dynamics are finished\\
					\hline
					\texttt{pspenergies.<name>} & single particle energies versus time, at time intervals commanded by input parameter jinfo\\
					\hline
					\texttt{pspvariances.<name>} & single particle energy variances versus time, at time intervals commanded by input parameter jinfo\\
					\hline
					\texttt{pspvariancesp.<name>} & single particle energy variances versus time (with correction by projection), at time intervals commanded by input parameter jinfo \\
					\hline
					\texttt{ptempion.<name>} & ion temperatures during ionic-core relaxation\\
					\hline
					\texttt{pvelion.<name>} & ion velocities during ionic-core relaxation, or dynamic calculation with molecular dynamics\\
					\hline
					\texttt{rsave.<name>} & This file contains all parameters of a static convergence to allow for a dynamic start without recomputing the statics: to use it set ismax=0 and istat=1  \\
					\hline
					\texttt{save.<name>} & This file contains all parameters to allow for a dynamic start at time : to use it set  irest>=0 \\
					\hline
					\texttt{Time} & Number of points in the calculation box and used wall time to complete the given number of iterations\\
					\hline
				\end{tabular}
			\end{table}

		\subsection{Results and observables}
			
	\section{Example dynamic calculations}
		\subsection{Exciting plasmon modes by applying a LASER boost}
		\subsection{Excite electrons with a laser pulse}
	
	\section{Electronic relaxation with RTA}
		\subsection{Additional \texttt{DYNAMIC} input parameters concerning RTA}

			\begin{table}[t]
				\caption{Parameters that control RTA in the DYNAMIC namelist in \texttt{for005.<name>}}\label{tab:dyn-input-params-rta}
				\begin{tabular}{|p{3.5cm}|p{11.2cm}|}
					\hline
					\multicolumn{2}{|c|}{The \texttt{DYNAMIC} namelist}\\
					\hline
					\multicolumn{2}{|c|}{\textit{\color{activeColor}flags for RTA}} \\
					\hline			
					\texttt{jrtaint} &  Modulus for calling the RTA subroutine, i.e., nr. of TDLDA steps per one RTA step. Course time step $\Delta t$ for RTA and fine time step for TDLDA \texttt{dt1} are related as $\Delta t=$\texttt{jrtaint}$*$\texttt{dt1}.\\
					\hline
					\texttt{rtamu} & Parameter $\mu$ in front of the quadratic density constraint in the DCMF Hamiltonian (\ref{eq:hDCMF})\\
					\hline
					\texttt{rtamuj} & Parameter $\mu_j$ in front of the quadratic current constraint in the DCMF Hamiltonian (\ref{eq:hDCMF})\\
					\hline
					\texttt{rtasumvar2max} & Termination criterion $\epsilon_0$ in the RTA step as used in figure \ref{fig:summaryDCMF}\\
					\hline
					\texttt{rtaeps} & Step size $\delta$ in the damping operator (cross ref to be defined) $\mathcal{D}$ for the RTA step.\\
					\hline
					\texttt{rtae0dmp} & Energy offset $E_{00}$ in the damping operator (cross ref to be defined) $\mathcal{D}$ for the RTA step\\
					\hline
					\texttt{rtasigee} & In medium $e^--e^-$ cross section used for the relaxation time (\ref{eq:relaxtime})\\
					\hline
					\texttt{rtars} & Effective Wigner-Seitz radius $r_s$ used for the relaxation time (\ref{eq:relaxtime})\\
					\hline
					\texttt{rtatempinit} & The value \texttt{rtatempinit}/10 is used as lower value for the search of temperature in \texttt{SUBROUTINE ferm1}\\
					\hline
				\end{tabular}
			\end{table}

		\subsection{Output files}
			\begin{table}[t]
				\caption{Output files that are specific to a RTA-enabled calculation}\label{tab:dynamic-output-files-rta}
				\begin{tabular}{|p{4.5cm}|p{10.2cm}|}
					\hline
					\texttt{convergenceRTA} & this file contains a log of the RTA iterations that contains information on the convergence details\\
					\hline
					\texttt{peqstate} &  parameters for convergence of the dtmf process: current iteration number, cycles to convergence, variance, residual err. on density, residual err. on current, parameters mu, muj, energy achieved\\
					\hline
					\texttt{prta} & prints at each RTA  step: time, entropy, laser energy and the mu and temperature of a fermi distribution fitted to the occupation numbers\\
					\hline
					\texttt{pspeed.<name>} & prints at each RTA step, along x axis, the reference density (spin up and down), achieved density (spin up and down), target x current, achieved x current \\
					\hline
				\end{tabular}
			\end{table}
			
			
			
								
%	\section{General thoughts and suggestions}
%	Some interesting test cases could be
%	\begin{itemize}
%		\item H$_2$O: Non-trivial molecule
%		\item Na$_7$: this is an interesting case because instead of preferring to be in a tetrahedral configuration, it prefers to be in a planer configuration.
%		\item Na$_{12}$: this is the first spin-polarised (has to be checked) cluster consisting of an even number of electrons.
%		\item The carbon chains C$_3$, C$_5$ and C$_7$: they all exhibit both organic- and metallic properties and have well defined plasmon modes. Furthermore, since these are all 1D and planar molecules, they can be calculated in a 1D or 2D box which makes the calculation much faster.
%		\item Na$_{93}$, : This is a magic number too.
%	\end{itemize}
	
%	\newpage
%	Document default font size set to 12~pt.\\
%	\tiny{tiny (\showfontsize)}\\
%	\scriptsize{scriptsize (\showfontsize)}\\
%	\footnotesize{footnotesize (\showfontsize)}\\
%	\small{small (\showfontsize)}\\
%	default (\showfontsize) (no size option)\\
%	\normalsize{normalsize (\showfontsize)}\\
%	\large{large (\showfontsize)}\\
%	\Large{Large (\showfontsize)}\\
%	\LARGE{LARGE (\showfontsize)}\\
%	\huge{huge (\showfontsize)}\\
%	\Huge{Huge (\showfontsize)}\\
%
%	\tiny{\sffamily tiny (\showfontsize)}\\
%	\scriptsize{\sffamily scriptsize (\showfontsize)}\\
%	\footnotesize{\sffamily footnotesize (\showfontsize)}\\
%	\small{\sffamily small (\showfontsize)}\\
%	\textsf{default} (\showfontsize) (in absence of a size option)\\
%	\normalsize{\sffamily normalsize (\showfontsize)}\\
%	\large{\sffamily large (\showfontsize)}\\
%	\Large{\sffamily Large (\showfontsize)}\\
%	\LARGE{\sffamily LARGE (\showfontsize)}\\
%	\huge{\sffamily huge (\showfontsize)}\\
%	\Huge{\sffamily Huge (\showfontsize)}\\


\end{document}
